\documentclass[letterpaper]{book}
\usepackage[times,hyper]{Rd}
\usepackage{makeidx}
\makeindex{}
\topmargin -0.25in \oddsidemargin 0in \evensidemargin 0in
\textheight 9in \textwidth 6.5in
\begin{document}
\setcounter{page}{89}
\chapter*{}
\begin{center}
{\textbf{\huge Package `PBSmodelling'}}
\par\bigskip{\large \today}
\end{center}
\begin{description}
\raggedright{}
\item[Version]\AsIs{2.11}
\item[Date]\AsIs{2009-06-08}
\item[Title]\AsIs{PBS Modelling 2.11}
\item[Author]\AsIs{Jon T. Schnute <Jon.Schnute@dfo-mpo.gc.ca>,
Alex Couture-Beil <alex@mofo.ca>,
Rowan Haigh <Rowan.Haigh@dfo-mpo.gc.ca>, and
Anisa Egeli <aegeli@gmail.com>}
\item[Maintainer]\AsIs{Jon Schnute <Jon.Schnute@dfo-mpo.gc.ca>}
\item[Depends]\AsIs{R (>= 2.7.0)}
\item[Suggests]\AsIs{PBSmapping, PBSddesolve, deSolve, BRugs, KernSmooth}
\item[Description]\AsIs{PBS Modelling provides software to facilitate the design,
testing, and operation of computer models. It focuses particularly on
tools that make it easy to construct and edit a customized graphical
user interface (GUI). Although it depends heavily on the R interface
to the Tcl/Tk package, a user does not need to know Tcl/Tk. The
package contains examples that illustrate models built with other R
packages, including PBSmapping, deSolve, PBSddesolve, and BRugs.
It also serves as a convenient prototype for building new R packages,
along with instructions and batch files to facilitate that process.
The R directory '.../library/PBSmodelling/doc' includes a complete user
guide PBSmodelling-UG.pdf. To use this package effectively,
please consult the guide.}
\item[License]\AsIs{GPL (>= 2)}
\end{description}
\Rdcontents{\R{} topics documented:}
\HeaderA{addArrows}{Add Arrows to a Plot Using Relative (0:1) Coordinates}{addArrows}
\keyword{iplot}{addArrows}
\begin{Description}\relax
Call the \code{arrows} function using relative (0:1) coordinates.
\end{Description}
\begin{Usage}
\begin{verbatim}
addArrows(x1, y1, x2, y2, ...)
\end{verbatim}
\end{Usage}
\begin{Arguments}
\begin{ldescription}
\item[\code{x1}] x-coordinate (0:1) at base of arrow.
\item[\code{y1}] y-coordinate (0:1) at base of arrow.
\item[\code{x2}] x-coordinate (0:1) at tip of arrow.
\item[\code{y2}] y-coordinate (0:1) at tip of arrow.
\item[\code{...}] additional parameters for the function \code{arrows}.
\end{ldescription}
\end{Arguments}
\begin{Details}\relax
Lines will be drawn from \code{(x1[i],y1[i])} to \code{(x2[i],y2[i])}
\end{Details}
\begin{Author}\relax
Jon Schnute, Pacific Biological Station, Nanaimo BC
\end{Author}
\begin{SeeAlso}\relax
\code{\LinkA{addLabel}{addLabel}}, \code{\LinkA{addLegend}{addLegend}}
\end{SeeAlso}
\begin{Examples}
\begin{ExampleCode}
tt=seq(from=-5,to=5,by=0.01)
plot(sin(tt), cos(tt)*(1-sin(tt)), type="l")
addArrows(0.2,0.5,0.8,0.5)
addArrows(0.8,0.95,0.95,0.55, col="#FF0066")
\end{ExampleCode}
\end{Examples}

\HeaderA{addLabel}{Add a Label to a Plot Using Relative (0:1) Coordinates}{addLabel}
\keyword{iplot}{addLabel}
\begin{Description}\relax
Place a label in a plot using relative (0:1) coordinates
\end{Description}
\begin{Usage}
\begin{verbatim}
addLabel(x, y, txt, ...)
\end{verbatim}
\end{Usage}
\begin{Arguments}
\begin{ldescription}
\item[\code{x}] x-axis coordinate in the range (0:1); can step outside.
\item[\code{y}] y-axis coordinate in the range (0:1); can step outside.
\item[\code{txt}] desired label at (\code{x,y}).
\item[\code{...}] additional arguments passed to the function \code{text}.
\end{ldescription}
\end{Arguments}
\begin{Author}\relax
Jon Schnute, Pacific Biological Station, Nanaimo BC
\end{Author}
\begin{SeeAlso}\relax
\code{\LinkA{addArrows}{addArrows}}, \code{\LinkA{addLegend}{addLegend}}
\end{SeeAlso}
\begin{Examples}
\begin{ExampleCode}
resetGraph()
addLabel(0.75,seq(from=0.9,to=0.1,by=-0.10),c('a','b','c'), col="#0033AA")
\end{ExampleCode}
\end{Examples}

\HeaderA{addLegend}{Add a Legend to a Plot Using Relative (0:1) Coordinates}{addLegend}
\keyword{iplot}{addLegend}
\begin{Description}\relax
Place a legend in a plot using relative (0:1) coordinates.
\end{Description}
\begin{Usage}
\begin{verbatim}
addLegend(x, y, ...) 
\end{verbatim}
\end{Usage}
\begin{Arguments}
\begin{ldescription}
\item[\code{x}] x-axis coordinate in the range (0:1); can step outside.
\item[\code{y}] y-axis coordinate in the range (0:1); can step outside.
\item[\code{...}] arguments used by the function \code{legend}, 
such as \code{lines}, \code{text}, or \code{rectangle}.
\end{ldescription}
\end{Arguments}
\begin{Author}\relax
Jon Schnute, Pacific Biological Station, Nanaimo BC
\end{Author}
\begin{SeeAlso}\relax
\code{\LinkA{addArrows}{addArrows}}, \code{\LinkA{addLabel}{addLabel}}
\end{SeeAlso}
\begin{Examples}
\begin{ExampleCode}
resetGraph(); n <- sample(1:length(colors()),15); clrs <- colors()[n]
addLegend(.2,1,fill=clrs,leg=clrs,cex=1.5)
\end{ExampleCode}
\end{Examples}

\HeaderA{calcFib}{Calculate Fibonacci Numbers by Several Methods}{calcFib}
\keyword{arith}{calcFib}
\begin{Description}\relax
Compute Fibonacci numbers using four different methods:
1) iteratively using R code,
2) via the closed function in R code,
3) iteratively in C using the \code{.C} function,
and 4) iteratively in C using the \code{.Call} function.
\end{Description}
\begin{Usage}
\begin{verbatim}
calcFib(n, len=1, method="C")
\end{verbatim}
\end{Usage}
\begin{Arguments}
\begin{ldescription}
\item[\code{n}] nth fibonacci number to calculate
\item[\code{len}] a vector of length \code{len} showing previous fibonacci numbers
\item[\code{method}] select method to use: \code{C}, \code{Call}, \code{R}, \code{closed}
\end{ldescription}
\end{Arguments}
\begin{Value}
Vector of the last \code{len} Fibonacci numbers 
calculated.
\end{Value}
\begin{Author}\relax
Jon Schnute, Pacific Biological Station, Nanaimo BC
\end{Author}

\HeaderA{calcGM}{Calculate the Geometric Mean, Allowing for Zeroes}{calcGM}
\keyword{arith}{calcGM}
\begin{Description}\relax
Calculate the geometric mean of a numeric 
vector, possibly excluding zeroes and/or adding an offset 
to compensate for zero values.
\end{Description}
\begin{Usage}
\begin{verbatim}calcGM(x, offset = 0, exzero = TRUE)\end{verbatim}
\end{Usage}
\begin{Arguments}
\begin{ldescription}
\item[\code{x}] vector of numbers
\item[\code{offset}] value to add to all components, including zeroes
\item[\code{exzero}] if \code{TRUE}, exclude zeroes (but still add the offset)
\end{ldescription}
\end{Arguments}
\begin{Value}
Geometric mean of the modified vector \code{x + offset}
\end{Value}
\begin{Note}\relax
\code{NA} values are automatically removed from \code{x}
\end{Note}
\begin{Author}\relax
Rowan Haigh, Pacific Biological Station, Nanaimo BC
\end{Author}
\begin{Examples}
\begin{ExampleCode}
calcGM(c(0,1,100))
calcGM(c(0,1,100),offset=0.01,exzero=FALSE)
\end{ExampleCode}
\end{Examples}

\HeaderA{calcMin}{Calculate the Minimum of a User-Defined Function}{calcMin}
\keyword{nonlinear}{calcMin}
\keyword{optimize}{calcMin}
\begin{Description}\relax
Minimization based on the R-stat functions \code{nlm}, \code{nlminb}, and \code{optim}.
Model parameters are scaled and can be active or not in the minimization.
\end{Description}
\begin{Usage}
\begin{verbatim}
calcMin(pvec, func, method="nlm", trace=0, maxit=1000, reltol=1e-8,
        steptol=1e-6, temp=10, repN=0, ...)
\end{verbatim}
\end{Usage}
\begin{Arguments}
\begin{ldescription}
\item[\code{pvec}] Initial values of the model parameters to be optimized.
\code{pvec} is a data frame comprising four columns (
\code{"val","min","max","active"}) and as many rows as there are model
parameters. The \code{"active"} field (logical) determines whether the 
parameters are estimated (\code{T}) or remain fixed (\code{F}).
\item[\code{func}] The user-defined function to be minimized (or maximized).
The function should return a scalar result.
\item[\code{method}] The minimization method to use: one of \code{nlm}, \code{nlminb},
\code{Nelder-Mead}, \code{BFGS}, \code{CG}, \code{L-BFGS-B}, or 
\code{SANN}. Default is \code{nlm}.
\item[\code{trace}] Non-negative integer. If positive, tracing information on the
progress of the minimization is produced. Higher values may produce more
tracing information: for method \code{"L-BFGS-B"} there are six levels of
tracing. Default is \code{0}.
\item[\code{maxit}] The maximum number of iterations. Default is \code{1000}.
\item[\code{reltol}] Relative convergence tolerance. The algorithm stops if it is
unable to reduce the value by a factor of \code{reltol*(abs(val)+reltol)}
at a step. Default is \code{1e-8}.
\item[\code{steptol}] A positive scalar providing the minimum allowable relative step length.
Default is \code{1e-6}.
\item[\code{temp}] Temperature controlling the \code{"SANN"} method. It is the
starting temperature for the cooling schedule. Default is \code{10}.
\item[\code{repN}] Reports the parameter and objective function values on the R-console
every \code{repN} evaluations. Default is \code{0} for no reporting.
\item[\code{...}] Further arguments to be passed to the optimizing function chosen:
\code{nlm}, \code{nlminb}, or \code{optim}.
Beware of partial matching to earlier arguments.
\end{ldescription}
\end{Arguments}
\begin{Details}\relax
See \code{optim} for details on the following methods: \code{Nelder-Mead},
\code{BFGS}, \code{CG}, \code{L-BFGS-B}, and \code{SANN}.
\end{Details}
\begin{Value}
A list with components:
\begin{ldescription}
\item[\code{Fout}] The output list from the optimizer function chosen through \code{method}.
\item[\code{iters}] Number of iterations.
\item[\code{evals}] Number of evaluations.
\item[\code{cpuTime}] The user CPU time to execute the minimization.
\item[\code{elapTime}] The total elapsed time to execute the minimization.
\item[\code{fminS}] The objective function value calculated at the start of the minimization.
\item[\code{fminE}] The objective function value calculated at the end of the minimization.
\item[\code{Pstart}] Starting values for the model parameters.
\item[\code{Pend}] Final values estimated for the model parameters from the minimization.
\item[\code{AIC}] Akaike's Information Criterion
\item[\code{message}] Convergence message from the minimization routine.
\end{ldescription}
\end{Value}
\begin{Note}\relax
Some arguments to \code{calcMin} have no effect depending on the \code{method} chosen.
\end{Note}
\begin{Author}\relax
Jon Schnute, Pacific Biological Station, Nanaimo BC
\end{Author}
\begin{SeeAlso}\relax
\code{\LinkA{scalePar}{scalePar}}, \code{\LinkA{restorePar}{restorePar}}, \code{\LinkA{calcMin}{calcMin}}, \code{\LinkA{GT0}{GT0}} \\
In the \code{stats} package: \code{nlm}, \code{nlminb}, and \code{optim}.
\end{SeeAlso}
\begin{Examples}
\begin{ExampleCode}
Ufun <- function(P) {
        Linf <- P[1]; K <- P[2]; t0 <- P[3]; obs <- afile$len;
        pred <- Linf * (1 - exp(-K*(afile$age-t0)));
        n <- length(obs); ssq <- sum((obs-pred)^2 );
        return(n*log(ssq)); };
afile <- data.frame(age=1:16,len=c(7.36,14.3,21.8,27.6,31.5,35.3,39,
        41.1,43.8,45.1,47.4,48.9,50.1,51.7,51.7,54.1));
pvec <- data.frame(val=c(70,0.5,0),min=c(40,0.01,-2),max=c(100,2,2),
        active=c(TRUE,TRUE,TRUE),row.names=c("Linf","K","t0"),
        stringsAsFactors=FALSE);
alist <- calcMin(pvec=pvec,func=Ufun,method="nlm",steptol=1e-4,repN=10);
print(alist[-1]); P <- alist$Pend;
resetGraph(); expandGraph();
xnew <- seq(afile$age[1],afile$age[nrow(afile)],len=100);
ynew <- P[1] * (1 - exp(-P[2]*(xnew-P[3])) );
plot(afile); lines(xnew,ynew,col="red",lwd=2); 
addLabel(.05,.88,paste(paste(c("Linf","K","t0"),round(P,c(2,4,4)),
        sep=" = "),collapse="\n"),adj=0,cex=0.9);
\end{ExampleCode}
\end{Examples}

\HeaderA{CCA.qbr}{Data: Sampled Counts of Quillback Rockfish 
(Sebastes maliger)}{CCA.qbr}
\keyword{datasets}{CCA.qbr}
\begin{Description}\relax
Count of sampled fish-at-age for quillback rockfish 
(\emph{Sebastes maliger}) in Johnstone Strait, British Columbia, 
from 1984 to 2004.
\end{Description}
\begin{Usage}
\begin{verbatim}data(CCA.qbr)\end{verbatim}
\end{Usage}
\begin{Format}\relax
A matrix with 70 rows (ages) and 14 columns 
(years). Attributes \dQuote{syrs} and \dQuote{cyrs} 
specify years of survey and commercial data, respectively.

\Tabular{ll}{
\code{[,c(3:5,9,13,14)]} & Counts-at-age from research
survey samples \\
\code{[,c(1,2,6:8,10:12)]} & Counts-at-age from commercial
fishery samples \\
}

All elements represent sampled counts-at-age in year. Zero-value
entries indicate no observations.
\end{Format}
\begin{Details}\relax
Handline surveys for rockfish have been conducted in 
Johnstone Strait (British Columbia) and adjacent waterways 
(126\eqn{^\circ}{}37'W to 126\eqn{^\circ}{}53'W, 
50\eqn{^\circ}{}32'N to 50\eqn{^\circ}{}39'N) since 1986. 
Yamanaka and Richards (1993) describe surveys conducted in 1986, 
1987, 1988, and 1992. In 2001, the Rockfish Selective Fishery 
Study (Berry 2001) targeted quillback rockfish \emph{Sebastes 
maliger} for experiments on improving survival after capture  by 
hook and line gear. The resulting data subsequently have been 
incorporated into the survey data series. The most recent survey 
in 2004 essentially repeated the 1992 survey design. Fish samples 
from surveys have been supplemented by commercial handline 
fishery samples taken from a larger region 
(126\eqn{^\circ}{}35'W to 127\eqn{^\circ}{}39'W, 
50\eqn{^\circ}{}32'N to 50\eqn{^\circ}{}59'N) in the years 
1984-1985, 1989-1991, 1993, 1996, and 2000 (Schnute and Haigh 
2007).
\end{Details}
\begin{Note}\relax
Years 1994, 1997-1999, and 2002-2003 do not have data.
\end{Note}
\begin{Source}\relax
Fisheries and Oceans Canada - GFBio database: \\
\url{http://www-sci.pac.dfo-mpo.gc.ca/sa-mfpd/statsamp/StatSamp_GFBio.htm}
\end{Source}
\begin{References}\relax
Berry, M.D. (2001) \emph{Area 12 (Inside) Rockfish Selective Fishery 
Study}. Science Council of British Columbia, Project Number \bold{FS00-05}.

Schnute, J.T. and Haigh, R. (2007) Compositional analysis of 
catch curve data with an application to \emph{Sebastes maliger}. 
\emph{ICES Journal of Marine Science} \bold{64}, 218--233.

Yamanaka, K.L. and Richards, L.J. (1993) 1992 Research catch and 
effort data on nearshore reef-fishes in British Columbia 
Statistical Area 12. \emph{Canadian Manuscript Report of Fisheries and 
Aquatic Sciences} \bold{2184}, 77 pp.
\end{References}
\begin{Examples}
\begin{ExampleCode}
# Plot age proportions (blue bubbles = survey data, red = commercial)
data(CCA.qbr); clrs=c("cornflowerblue","orangered")
z <- CCA.qbr; cyr <- attributes(z)$cyrs;
z <- apply(z,2,function(x){x/sum(x)}); z[,cyr] <- -z[,cyr];
x <- as.numeric(dimnames(z)[[2]]); xlim <- range(x) + c(-.5,.5);
y <- as.numeric(dimnames(z)[[1]]); ylim <- range(y) + c(-1,1);
expandGraph(mgp=c(2,.5,0),las=1)
plotBubbles(z,xval=x,yval=y,powr=.5,size=0.15,clrs=clrs,
   xlim=xlim,ylim=ylim,xlab="Year",ylab="Age",cex.lab=1.5)
addLegend(.5,1,bty="n",pch=1,cex=1.2,col=clrs,
   legend=c("Survey","Commercial"),horiz=TRUE,xjust=.5)
\end{ExampleCode}
\end{Examples}

\HeaderA{chooseWinVal}{Choose and Set a String Item in a GUI}{chooseWinVal}
\keyword{device}{chooseWinVal}
\keyword{utilities}{chooseWinVal}
\begin{Description}\relax
Prompts the user to choose one string item from a list of 
choices displayed in a GUI, then sets a specified variable in 
a target GUI.
\end{Description}
\begin{Usage}
\begin{verbatim}
chooseWinVal(choice, varname, winname="window")
\end{verbatim}
\end{Usage}
\begin{Arguments}
\begin{ldescription}
\item[\code{choice}] vector of strings from which to choose 
\item[\code{varname}] variable name to which \code{choice} is assigned in the target GUI 
\item[\code{winname}] window name for the target GUI 
\end{ldescription}
\end{Arguments}
\begin{Details}\relax
\code{chooseWinVal} activates a \code{setWinVal} command through an
\code{onClose} function created by the \code{getChoice} command and 
modified by \code{chooseWinVal}.
\end{Details}
\begin{Value}
No value is returned directly. The choice is written to the PBS options 
workspace, accessible through \\ \code{getPBSoptions("getChoice")}. Also set 
in PBS options is the window name from which the choice was activated.
\end{Value}
\begin{Note}\relax
Microsoft Windows users may experience difficulties switching focus between the 
R console and GUI windows. The latter frequently disappear from the screen and 
need to be reselected (either clicking on the task bar or pressing \textless{}Alt\textgreater{}\textless{}Tab\textgreater{}. 
This issue can be resolved by switching from MDI to SDI mode. From the R console 
menu bar, select \textless{}Edit\textgreater{} and \textless{}GUI preferences\textgreater{}, then change the value of 
\dQuote{single or multiple windows} to SDI.
\end{Note}
\begin{Author}\relax
Rowan Haigh, Pacific Biological Station, Nanaimo BC
\end{Author}
\begin{SeeAlso}\relax
\code{\LinkA{getChoice}{getChoice}}, \code{\LinkA{getWinVal}{getWinVal}}, \code{\LinkA{setWinVal}{setWinVal}}
\end{SeeAlso}
\begin{Examples}
\begin{ExampleCode}
## Not run: 
dfnam <-
        c("airquality","attitude","ChickWeight","faithful","freeny",
        "iris","LifeCycleSavings","longley","morley","Orange",
        "quakes","randu","rock","stackloss","swiss","trees")

wlist <- c(
        "window name=choisir title=\"Test chooseWinVal\"",
        "label text=\"Press <ENTER> in the green entry box
        \nto choose a file, then press <GO>\" sticky=W pady=5",
        "grid 1 3 sticky=W",
        "label text=File: sticky=W",
        "entry name=fnam mode=character width=23 value=\"\" 
        func=chFile entrybg=darkolivegreen1 pady=5",
        "button text=GO bg=green sticky=W func=test",
        "")

chFile <- function(ch=dfnam,fn="fnam") 
        {chooseWinVal(ch,fn,winname="choisir")};

#-- Example 1 GUI test
test <- function() {
        getWinVal(winName="choisir",scope="L")
        if (fnam!="" && any(fnam==dfnam)) {
                file <- get(fnam);
                pairs(file,gap=0); }
        else {
                resetGraph(); 
                addLabel(.5,.5,"Press <ENTER> in the green entry box
                \nto choose a file, then press <GO>", col="red",cex=1.5)}};

#-- Example 2 Non-GUI test
#To try the non-GUI version, type 'test2()' on the command line
test2 <- function(fnames=dfnam) {
  frame();resetGraph()
  again <- TRUE;
  while (again) {
    fnam <- sample(fnames,1); file <- get(fnam); 
    flds <- names(file);
    xfld <- getChoice(paste("Pick x-field from",fnam),flds,gui=F);
    yfld <- getChoice(paste("Pick y-field from",fnam),flds,gui=F)
    plot(file[,xfld],file[,yfld],xlab=xfld,ylab=yfld,
      pch=16,cex=1.2,col="red");
    again <- getChoice("Plot another pair?",gui=F) }
  }
require(PBSmodelling)
createWin(wlist,astext=T); test();
## End(Not run)
\end{ExampleCode}
\end{Examples}

\HeaderA{cleanProj}{Launch a GUI for Project File Deletion}{cleanProj}
\keyword{utilities}{cleanProj}
\begin{Description}\relax
Launches a new window which contains an interface for deleting 
junk files associated with a prefix and a set of suffixes 
(e.g., PBSadmb project) from the working directory.
\end{Description}
\begin{Usage}
\begin{verbatim}
cleanProj(prefix, suffix, files)
\end{verbatim}
\end{Usage}
\begin{Arguments}
\begin{ldescription}
\item[\code{prefix}] default prefix for file names.
\item[\code{suffix}] character vector of suffixes used for clean options.
\item[\code{files}] character vector of file names used for clean options.
\end{ldescription}
\end{Arguments}
\begin{Details}\relax
All arguments may contain wildcard characters (\code{"*"} to match 0 or
more characters, \code{"?"} to match any single character).

The GUI includes the following:
\Tabular{ll}{
\bold{1} & An entry box for the prefix. \\
& The default value of this entry box is taken from \code{prefix}. \\
\bold{2} & Check boxes for each suffix in the \code{suffix} argument and \\
& for each file name in the \code{files} argument.\\
\bold{3} & Buttons marked "Select All" and "Select None" for \\ 
& selecting and clearing all the check boxes, respectively.\\
\bold{4} & A "Clean" button that deletes files in the working directory \\
& matching one of the following criteria: \\
& (i) file name matches both an expansion of a concatenation of a \\
& prefix in the entry box and a suffix chosen with a check box; or \\
& (ii) file name matches an expansion of a file chosen with a check box.
}
\end{Details}
\begin{Author}\relax
Anisa Egeli, Vancouver Island University, Nanaimo BC
\end{Author}
\begin{Examples}
\begin{ExampleCode}
## Not run: 
cleanProj(prefix="foo",suffix=c(".a*",".b?",".c","-old.d"),files=c("red","blue"))
## End(Not run)
\end{ExampleCode}
\end{Examples}

\HeaderA{cleanWD}{Launch a GUI for File Deletion}{cleanWD}
\keyword{utilities}{cleanWD}
\begin{Description}\relax
Launches a new window which contains an interface for deleting 
specified files from the working directory.
\end{Description}
\begin{Usage}
\begin{verbatim}
cleanWD(files)
\end{verbatim}
\end{Usage}
\begin{Arguments}
\begin{ldescription}
\item[\code{files}] character vector of file names used for clean options.
\end{ldescription}
\end{Arguments}
\begin{Details}\relax
All arguments may contain wildcard characters (\code{"*"} to match 0 or
more characters, \code{"?"} to match any single character).

The GUI includes the following:
\Tabular{ll}{
\bold{1} & Check boxes for each suffix in the \code{suffix} argument and \\
& for each file name in the \code{files} argument.\\
\bold{2} & Buttons marked "Select All" and "Select None" for \\ 
& selecting and clearing all the check boxes, respectively.\\
\bold{3} & A "Clean" button that deletes files in the working directory \\
& matching file name expansion of files chosen with a check box.
}
\end{Details}
\begin{Author}\relax
Rowan Haigh, Pacific Biological Station, Nanaimo BC
\end{Author}
\begin{Examples}
\begin{ExampleCode}
## Not run: 
cleanWD(c("*.bak","*.tmp","junk*"))
## End(Not run)
\end{ExampleCode}
\end{Examples}

\HeaderA{clearAll}{Remove all R Objects From the Global Environment}{clearAll}
\keyword{methods}{clearAll}
\begin{Description}\relax
Generic function to clear all objects from .RData in R
\end{Description}
\begin{Usage}
\begin{verbatim}
clearAll(hidden=TRUE, verbose=TRUE, PBSsave=TRUE)
\end{verbatim}
\end{Usage}
\begin{Arguments}
\begin{ldescription}
\item[\code{hidden}] if \code{TRUE}, remove variables that start with a dot(.).
\item[\code{verbose}] if \code{TRUE}, report all removed items.
\item[\code{PBSsave}] if \code{TRUE}, do not remove \code{.PBSmod}.
\end{ldescription}
\end{Arguments}
\begin{Author}\relax
Jon Schnute, Pacific Biological Station, Nanaimo BC
\end{Author}

\HeaderA{clearPBSext}{Clear File Extension Associations}{clearPBSext}
\keyword{methods}{clearPBSext}
\begin{Description}\relax
Disassociate any number of file extensions from commands previously saved
with \code{setPBSext}.
\end{Description}
\begin{Usage}
\begin{verbatim}clearPBSext(ext)\end{verbatim}
\end{Usage}
\begin{Arguments}
\begin{ldescription}
\item[\code{ext}] optional character vector of file extensions to clear; if
unspecified, all associations are removed
\end{ldescription}
\end{Arguments}
\begin{Author}\relax
Alex Couture-Beil, Malaspina University-College, Nanaimo BC
\end{Author}
\begin{SeeAlso}\relax
\code{\LinkA{setPBSext}{setPBSext}}, \code{\LinkA{getPBSext}{getPBSext}}, \code{\LinkA{openFile}{openFile}}
\end{SeeAlso}

\HeaderA{clearRcon}{Clear the R Console Window}{clearRcon}
\aliasA{focusRcon}{clearRcon}{focusRcon}
\keyword{device}{clearRcon}
\begin{Description}\relax
Clear the R console window using a Visual Basic shell script.
\end{Description}
\begin{Usage}
\begin{verbatim}
clearRcon(os=.Platform$OS.type)
\end{verbatim}
\end{Usage}
\begin{Arguments}
\begin{ldescription}
\item[\code{os}] operating system (e.g., \code{"windows"}, \code{"unix"}). 
\end{ldescription}
\end{Arguments}
\begin{Details}\relax
Creates a VB shell script file called \code{clearRcon.vba} in R's temporary 
working directory, then executes the script using the \code{shell} command.

Similarly, \code{focusRcon()} gives the focus to the R console window by 
creating a Visual Basic shell script called \code{focusRgui.vba} in R's 
temporary working directory, then executes it using the \code{shell} command.

These commands will only work on Windows operating platforms, 
using the system's executable \\
\code{\%SystemRoot\%\bsl{}system32\bsl{}cscript.exe}.
\end{Details}
\begin{Author}\relax
Norm Olsen, Pacific Biological Station, Nanaimo BC
\end{Author}
\begin{SeeAlso}\relax
\code{\LinkA{cleanWD}{cleanWD}}, \code{\LinkA{clearPBSext}{clearPBSext}}, \code{\LinkA{clearWinVal}{clearWinVal}}
\end{SeeAlso}

\HeaderA{clearWinVal}{Remove all Current Widget Variables}{clearWinVal}
\keyword{methods}{clearWinVal}
\begin{Description}\relax
Remove all global variables that share a name in common with any widget variable name 
defined in \\ \code{names(getWinVal())}. Use this function with caution.
\end{Description}
\begin{Usage}
\begin{verbatim}clearWinVal()\end{verbatim}
\end{Usage}
\begin{Author}\relax
Alex Couture-Beil, Malaspina University-College, Nanaimo BC
\end{Author}
\begin{SeeAlso}\relax
\code{\LinkA{getWinVal}{getWinVal}}
\end{SeeAlso}

\HeaderA{clipVector}{Clip a Vector at One or Both Ends}{clipVector}
\keyword{data}{clipVector}
\keyword{utilities}{clipVector}
\begin{Description}\relax
Clip a vector at one or both ends using the specified clip 
pattern to match.
\end{Description}
\begin{Usage}
\begin{verbatim}
clipVector(vec, clip, end=0)
\end{verbatim}
\end{Usage}
\begin{Arguments}
\begin{ldescription}
\item[\code{vec}] vector object to clip
\item[\code{clip}] value or string specifying repeated values to clip from ends
\item[\code{end}] end to clip \code{clip} from: 0=both, 1=front, 2=back
\end{ldescription}
\end{Arguments}
\begin{Details}\relax
If the vector is named, the names are retained. Otherwise,
element positions are assigned as the vector's names.
\end{Details}
\begin{Value}
Clipped vector with names.
\end{Value}
\begin{Author}\relax
Rowan Haigh, Pacific Biological Station, Nanaimo BC
\end{Author}
\begin{SeeAlso}\relax
\code{\LinkA{createVector}{createVector}}
\end{SeeAlso}
\begin{Examples}
\begin{ExampleCode}
x=c(0,0,0,0,1,1,1,1,0,0)
clipVector(x,0)

x=c(TRUE,TRUE,FALSE,TRUE)
clipVector(x,TRUE)

x=c("red","tide","red","red")
clipVector(x,"red",2)
\end{ExampleCode}
\end{Examples}

\HeaderA{closeWin}{Close GUI Window(s)}{closeWin}
\keyword{utilities}{closeWin}
\begin{Description}\relax
Close (destroy) one or more windows made with \code{createWin}.
\end{Description}
\begin{Usage}
\begin{verbatim}closeWin(name)\end{verbatim}
\end{Usage}
\begin{Arguments}
\begin{ldescription}
\item[\code{name}] a vector of window names that indicate which windows to close. These 
names appear in the \emph{window description file}(s) on the line(s) defining WINDOW widgets.
If \code{name} is omitted, all active windows will be closed.
\end{ldescription}
\end{Arguments}
\begin{Author}\relax
Alex Couture-Beil, Malaspina University-College, Nanaimo BC
\end{Author}
\begin{SeeAlso}\relax
\code{\LinkA{createWin}{createWin}}
\end{SeeAlso}

\HeaderA{compileC}{Compile a C File into a Shared Library Object}{compileC}
\keyword{programming}{compileC}
\keyword{interface}{compileC}
\begin{Description}\relax
This function provides an alternative to using R's \code{SHLIB} 
command to compile C code into a shared library object.
\end{Description}
\begin{Usage}
\begin{verbatim}
compileC(file, lib="", options="", logWindow=TRUE, logFile=TRUE)
\end{verbatim}
\end{Usage}
\begin{Arguments}
\begin{ldescription}
\item[\code{file}] name of the file to compile.
\item[\code{lib}] name of shared library object (without extension).
\item[\code{options}] linker options (in one string) to prepend to a compilation command.
\item[\code{logWindow}] if \code{TRUE}, a log window containing the compiler output will be displayed.
\item[\code{logFile}] if \code{TRUE}, a log file containing the compiler output will be created.
\end{ldescription}
\end{Arguments}
\begin{Details}\relax
If \code{lib=""}, it will take the same name as \code{file} (with a different extension).

If an object with the same name has already been dynamically loaded in R, 
it will be unloaded automatically for recompilation.

The name of the log file, if created, uses the string value from \code{lib} 
concatenated with \code{".log"}.
\end{Details}
\begin{Author}\relax
Anisa Egeli, Vancouver Island University, Nanaimo BC
\end{Author}
\begin{SeeAlso}\relax
\code{\LinkA{loadC}{loadC}}
\end{SeeAlso}
\begin{Examples}
\begin{ExampleCode}
## Not run: 
compileC("myFile.c", lib="myLib", options="myObj.o")
## End(Not run)
\end{ExampleCode}
\end{Examples}

\HeaderA{compileDescription}{Convert and Save a Window Description as a List}{compileDescription}
\keyword{utilities}{compileDescription}
\begin{Description}\relax
Convert a \emph{window description file} (ASCII markup file) to an equivalent 
\emph{window description list}. The output list (an ASCII file containing R-source code) 
is complete, i.e., all default values have been added.
\end{Description}
\begin{Usage}
\begin{verbatim}compileDescription(descFile, outFile)\end{verbatim}
\end{Usage}
\begin{Arguments}
\begin{ldescription}
\item[\code{descFile}] name of \emph{window description file} (markup file).
\item[\code{outFile}] name of output file containing R source code.
\end{ldescription}
\end{Arguments}
\begin{Details}\relax
The \emph{window description file} \code{descFile} is converted to a list, 
which is then converted to R code, and saved to \code{outFile}.
\end{Details}
\begin{Author}\relax
Alex Couture-Beil, Malaspina University-College, Nanaimo BC
\end{Author}
\begin{SeeAlso}\relax
\code{\LinkA{parseWinFile}{parseWinFile}}, \code{\LinkA{createWin}{createWin}}
\end{SeeAlso}

\HeaderA{convSlashes}{Convert Slashes from UNIX to DOS}{convSlashes}
\keyword{character}{convSlashes}
\begin{Description}\relax
Convert slashes in a string from \samp{/} to \samp{\bsl{}\bsl{}} if the 
operating system is \samp{windows}. Do the reverse if the OS is
\samp{unix}.
\end{Description}
\begin{Usage}
\begin{verbatim}
convSlashes(expr, os=.Platform$OS.type, addQuotes=FALSE)
\end{verbatim}
\end{Usage}
\begin{Arguments}
\begin{ldescription}
\item[\code{expr}] String value (usually a system pathway). 
\item[\code{os}] operating system (either \code{"windows"} or \code{"unix"}). 
\item[\code{addQuotes}] logical: if \code{TRUE}, enclose the string expression
in escaped double quotation marks. 
\end{ldescription}
\end{Arguments}
\begin{Value}
Returns the input string modified to have the appropriate slashes for the 
specified operating system.
\end{Value}
\begin{Author}\relax
Rowan Haigh, Pacific Biological Station, Nanaimo BC
\end{Author}

\HeaderA{createVector}{Create a GUI with a Vector Widget}{createVector}
\keyword{utilities}{createVector}
\begin{Description}\relax
Create a basic window containing a vector and a submit button. 
This provides a quick way to create a window without the need 
for a \emph{window description file}.
\end{Description}
\begin{Usage}
\begin{verbatim}
createVector(vec, vectorLabels=NULL, func="", 
             windowname="vectorwindow")\end{verbatim}
\end{Usage}
\begin{Arguments}
\begin{ldescription}
\item[\code{vec}] a vector of strings representing widget variables. 
The values in \code{vec} become the default values for the widget.
If \code{vec} is named, the names are used as the variable names. 
\item[\code{vectorLabels}] an optional vector of strings to use as labels above each widget.
\item[\code{func}] string name of function to call when new data are entered 
in widget boxes or when "GO" is pressed.
\item[\code{windowname}] unique window name, required if multiple vector windows are created.
\end{ldescription}
\end{Arguments}
\begin{Author}\relax
Alex Couture-Beil, Malaspina University-College, Nanaimo BC
\end{Author}
\begin{SeeAlso}\relax
\code{\LinkA{createWin}{createWin}}
\end{SeeAlso}
\begin{Examples}
\begin{ExampleCode}
## Not run: 
#user defined function which is called on new data      
drawLiss <- function() {
  getWinVal(scope="L");
  tt <- 2*pi*(0:k)/k; x <- sin(2*pi*m*tt); y <- sin(2*pi*(n*tt+phi));
  plot(x,y,type="p"); invisible(NULL); };

#create the vector window
createVector(c(m=2, n=3, phi=0, k=1000), 
  vectorLabels=c("x cycles","y cycles", "y phase", "points"), 
  func="drawLiss");
## End(Not run)
\end{ExampleCode}
\end{Examples}

\HeaderA{createWin}{Create a GUI Window}{createWin}
\keyword{utilities}{createWin}
\begin{Description}\relax
Create a GUI window with widgets using instructions from a 
Window Description (markup) File.
\end{Description}
\begin{Usage}
\begin{verbatim}createWin(fname, astext=FALSE)\end{verbatim}
\end{Usage}
\begin{Arguments}
\begin{ldescription}
\item[\code{fname}] name of \emph{window description file} 
or list returned from \code{parseWinFile}.
\item[\code{astext}] logical: if \code{TRUE}, interpret \code{fname} 
as a vector of strings with each element representing a line 
in a \emph{window description file}.
\end{ldescription}
\end{Arguments}
\begin{Details}\relax
Generally, the markup file contains a single widget per line. However, widgets 
can span multiple lines by including a backslash ('\bsl{}') character at the end of 
a line, prompting the suppression of the newline character.

For more details on widget types and markup file, see \dQuote{PBSModelling-UG.pdf} 
in the R directory \\ \code{.../library/PBSmodelling/doc}.

It is possible to use a Window Description List produced by 
\code{compileDescription} rather than a file name for \code{fname}.

Another alternative is to pass a vector of characters to \code{fname} and set 
\code{astext=T}. This vector represents the file contents where each element 
is equivalent to a new line in the \emph{window description file}.
\end{Details}
\begin{Note}\relax
Microsoft Windows users may experience difficulties switching focus between the 
R console and GUI windows. The latter frequently disappear from the screen and 
need to be reselected (either clicking on the task bar or pressing \textless{}Alt\textgreater{}\textless{}Tab\textgreater{}. 
This issue can be resolved by switching from MDI to SDI mode. From the R console 
menu bar, select \textless{}Edit\textgreater{} and \textless{}GUI preferences\textgreater{}, then change the value of 
\dQuote{single or multiple windows} to SDI.
\end{Note}
\begin{Author}\relax
Alex Couture-Beil, Malaspina University-College, Nanaimo BC
\end{Author}
\begin{SeeAlso}\relax
\code{\LinkA{parseWinFile}{parseWinFile}}, \code{\LinkA{getWinVal}{getWinVal}}, \code{\LinkA{setWinVal}{setWinVal}}

\code{\LinkA{closeWin}{closeWin}}, \code{\LinkA{compileDescription}{compileDescription}}, \code{\LinkA{createVector}{createVector}}

\code{\LinkA{initHistory}{initHistory}} for an example of using \code{astext=TRUE}
\end{SeeAlso}
\begin{Examples}
\begin{ExampleCode}
## Not run: 
# See file .../library/PBSmodelling/testWidgets/LissWin.txt

# Calculate and draw the Lissajous figure
drawLiss <- function() {
   getWinVal(scope="L"); ti=2*pi*(0:k)/k;
   x=sin(2*pi*m*ti);     y=sin(2*pi*(n*ti+phi));
   plot(x,y,type=ptype); invisible(NULL); };
createWin(system.file("testWidgets/LissWin.txt",package="PBSmodelling"));
## End(Not run)
\end{ExampleCode}
\end{Examples}

\HeaderA{declareGUIoptions}{Declare Option Names that Correspond with Widget Names}{declareGUIoptions}
\begin{Description}\relax
This function allows a GUI creator to specify widget names that 
correspond to names in PBS options. These widgets can then be 
used to load and set PBS options using \code{getGUIoptions} and 
\code{setGUIoptions}.
\end{Description}
\begin{Usage}
\begin{verbatim}
declareGUIoptions(newOptions)
\end{verbatim}
\end{Usage}
\begin{Arguments}
\begin{ldescription}
\item[\code{newOptions}] a character vector of option names
\end{ldescription}
\end{Arguments}
\begin{Details}\relax
\code{declareGUIoptions} is typically called in a GUI initialization function.
The option names are remembered and used for the functions 
\code{getGUIoptions}, \code{setGUIoptions}, and \code{promptSave}.
\end{Details}
\begin{Author}\relax
Anisa Egeli, Vancouver Island University, Nanaimo BC
\end{Author}
\begin{SeeAlso}\relax
\code{\LinkA{getGUIoptions}{getGUIoptions}}, \code{\LinkA{setGUIoptions}{setGUIoptions}},
\code{\LinkA{promptWriteOptions}{promptWriteOptions}}
\end{SeeAlso}
\begin{Examples}
\begin{ExampleCode}
## Not run: 
declareGUIOptions("editor")
## End(Not run)
\end{ExampleCode}
\end{Examples}

\HeaderA{doAction}{Execute Action Created by a Widget}{doAction}
\keyword{utilities}{doAction}
\keyword{character}{doAction}
\begin{Description}\relax
Executes the action expression formulated by the user and 
written as an \samp{action} by a widget.
\end{Description}
\begin{Usage}
\begin{verbatim}
doAction(act, envir=.GlobalEnv)
\end{verbatim}
\end{Usage}
\begin{Arguments}
\begin{ldescription}
\item[\code{act}] string representing an expression that can be executed
\item[\code{envir}] the R environment in which to evaluate the action; 
the default is the global environment or user's workspace. 
\end{ldescription}
\end{Arguments}
\begin{Details}\relax
If \code{act} is missing, \code{doAction} looks for it in the action
directory of the window's widget directory in \code{.PBSmod}. This 
action can be accessed through \code{getWinAct()[1]}.

Due to parsing complications, the expression \code{act} must contain 
the backtick character \samp{`} wherever there is to be an internal 
double quote \samp{"} character. For example,
\begin{alltt}"openFile(paste(getWinVal()\$prefix,`.tpl`,sep=``))"\end{alltt}
\end{Details}
\begin{Value}
Invisibly returns the string expression \code{act}.
\end{Value}
\begin{Author}\relax
Rowan Haigh, Pacific Biological Station, Nanaimo BC
\end{Author}

\HeaderA{drawBars}{Draw a Linear Barplot on the Current Plot}{drawBars}
\keyword{hplot}{drawBars}
\begin{Description}\relax
Draw a linear barplot on the current plot.
\end{Description}
\begin{Usage}
\begin{verbatim}drawBars(x, y, width, base = 0, ...)\end{verbatim}
\end{Usage}
\begin{Arguments}
\begin{ldescription}
\item[\code{x}] x-coordinates
\item[\code{y}] y-coordinates
\item[\code{width}] bar width, computed if missing
\item[\code{base}] y-value of the base of each bar
\item[\code{...}] further graphical parameters (see \code{par}) may also be supplied as arguments
\end{ldescription}
\end{Arguments}
\begin{Author}\relax
Jon Schnute, Pacific Biological Station, Nanaimo BC
\end{Author}
\begin{Examples}
\begin{ExampleCode}
plot(0:10,0:10,type="n")
drawBars(x=1:9,y=9:1,col="deepskyblue4",lwd=3)
\end{ExampleCode}
\end{Examples}

\HeaderA{evalCall}{Evaluate a Function Call}{evalCall}
\keyword{programming}{evalCall}
\keyword{character}{evalCall}
\begin{Description}\relax
Evaluates a function call after resolving potential
argument conflicts.
\end{Description}
\begin{Usage}
\begin{verbatim}
evalCall(fn, argu, ..., envir = parent.frame(),
    checkdef=FALSE, checkpar=FALSE)
\end{verbatim}
\end{Usage}
\begin{Arguments}
\begin{ldescription}
\item[\code{fn}] R function 
\item[\code{argu}] list of explicitly named arguments and their values to pass to \code{fn}. 
\item[\code{...}] additional arguments that a user might wish to pass to \code{fn}. 
\item[\code{envir}] environment from which the call originates (currently has no use or effect). 
\item[\code{checkdef}] logical: if \code{TRUE}, gather additional formal arguments from the 
functions default function. 
\item[\code{checkpar}] logical: if \code{TRUE}, gather additional graphical arguments from 
the list object \code{par}. 
\end{ldescription}
\end{Arguments}
\begin{Details}\relax
This function builds a call to the specified function and executes it. 
During the build, optional arguments \dots are checked for \\
(i) duplication with explicit arguments \code{argu}: if any are duplicated,
the user-supplied arguments supercede the explict ones; \\
(ii) availability as usuable arguments in \code{fn}, \code{fn.default} if
\code{checkdef=TRUE}, and \code{par} if \code{checkpar=TRUE}.
\end{Details}
\begin{Value}
Invisibly returns the string expression of the function call that is
passed to \code{eval(parse(text=expr))}.
\end{Value}
\begin{Author}\relax
Rowan Haigh, Pacific Biological Station, Nanaimo BC
\end{Author}
\begin{SeeAlso}\relax
\code{\LinkA{doAction}{doAction}}
\end{SeeAlso}

\HeaderA{expandGraph}{Expand the Plot Area by Adjusting Margins}{expandGraph}
\keyword{device}{expandGraph}
\begin{Description}\relax
Optimize the plotting region(s) by minimizing margins.
\end{Description}
\begin{Usage}
\begin{verbatim}
expandGraph(mar=c(4,3,1.2,0.5), mgp=c(1.6,.5,0),...)
\end{verbatim}
\end{Usage}
\begin{Arguments}
\begin{ldescription}
\item[\code{mar}] numerical vector of the form 'c(bottom, left, top, right)'
specifying the margins of the plot
\item[\code{mgp}] numerical vector of the form 'c(axis title, axis labels, axis line)'
specifying the margins for axis title, axis labels, and axis line
\item[\code{...}] additional graphical parameters to be passed to \code{par}
\end{ldescription}
\end{Arguments}
\begin{Author}\relax
Jon Schnute, Pacific Biological Station, Nanaimo BC
\end{Author}
\begin{SeeAlso}\relax
\code{\LinkA{resetGraph}{resetGraph}}
\end{SeeAlso}
\begin{Examples}
\begin{ExampleCode}
resetGraph(); expandGraph(mfrow=c(2,1));
tt=seq(from=-10, to=10, by=0.05);

plot(tt,sin(tt), xlab="this is the x label",  ylab="this is the y label", 
        main="main title", sub="sometimes there is a \"sub\" title")
plot(cos(tt),sin(tt*2), xlab="cos(t)", ylab="sin(2 t)", main="main title", 
        sub="sometimes there is a \"sub\" title")
\end{ExampleCode}
\end{Examples}

\HeaderA{exportHistory}{Export a Saved History}{exportHistory}
\keyword{list}{exportHistory}
\begin{Description}\relax
Export the current history list.
\end{Description}
\begin{Usage}
\begin{verbatim}
exportHistory(hisname="", fname="")
\end{verbatim}
\end{Usage}
\begin{Arguments}
\begin{ldescription}
\item[\code{hisname}] name of the history list to export. If set to \code{""}, 
the value from \code{getWinAct()[1]} will be used instead.
\item[\code{fname}] file name where history will be saved. If it is set to \code{""}, 
a \textless{}Save As\textgreater{} window will be displayed.
\end{ldescription}
\end{Arguments}
\begin{Author}\relax
Alex Couture-Beil, Malaspina University-College, Nanaimo BC
\end{Author}
\begin{SeeAlso}\relax
\code{\LinkA{importHistory}{importHistory}}, \code{\LinkA{initHistory}{initHistory}}, \code{\LinkA{promptSaveFile}{promptSaveFile}}
\end{SeeAlso}

\HeaderA{findPat}{Search a Character Vector to Find Multiple Patterns}{findPat}
\keyword{utilities}{findPat}
\begin{Description}\relax
Use all available patterns in \code{pat} to search in \code{vec}, 
and return the matched elements in \code{vec}.
\end{Description}
\begin{Usage}
\begin{verbatim}
findPat(pat, vec)
\end{verbatim}
\end{Usage}
\begin{Arguments}
\begin{ldescription}
\item[\code{pat}] character vector of patterns to match in \code{vec}
\item[\code{vec}] character vector where matches are sought
\end{ldescription}
\end{Arguments}
\begin{Value}
A character vector of all matched strings.
\end{Value}
\begin{Author}\relax
Rowan Haigh, Pacific Biological Station, Nanaimo BC
\end{Author}
\begin{Examples}
\begin{ExampleCode}
#find all strings with a vowel, or that start with a number
findPat(c("[aeoiy]", "^[0-9]"), c("hello", "WRLD", "11b"))
\end{ExampleCode}
\end{Examples}

\HeaderA{findPrefix}{Find a Prefix Based on Names of Existing Files}{findPrefix}
\keyword{file}{findPrefix}
\begin{Description}\relax
Find the prefixes of files with a given suffix in the working directory.
\end{Description}
\begin{Usage}
\begin{verbatim}
findPrefix(suffix)
\end{verbatim}
\end{Usage}
\begin{Arguments}
\begin{ldescription}
\item[\code{suffix}] character vector of suffixes
\end{ldescription}
\end{Arguments}
\begin{Details}\relax
The function \code{findPrefix} locates all files in the working directory that end with
one of the provided suffixes. The suffixes may contain wildcards (\code{"*"} to match 0
or more characters, \code{"?"} to match any single character).

If \code{findPrefix} was called from a widget as specified in a 
\emph{window description file}, then the value of a widget named 
\code{prefix} will be set to the prefix of the first matching file 
found, with an exception: if the value of the prefix widget
matches one of the file prefixes found, it will not be changed.

To use this function in a \emph{window description file}, the action of the widget is
used to specify the suffixes to match, with the suffixes separated by commas.
For example, \code{action=.c,.cpp} would set a prefix widget to the first file found
with an extension \code{.c} or \code{.cpp}.
\end{Details}
\begin{Value}
A character vector of all the prefixes of files in the working directory that
matched to one of the given suffixes.
\end{Value}
\begin{Author}\relax
Anisa Egeli, Vancouver Island University, Nanaimo BC
\end{Author}
\begin{SeeAlso}\relax
\code{\LinkA{setwdGUI}{setwdGUI}}
\end{SeeAlso}
\begin{Examples}
\begin{ExampleCode}
## Not run: 
# Match files that end with '.a' followed by 0 or more characters,
# '.b' followed by any single character, '.c', or '-old.d'
# (a suffix does not have to be a file extension)
findPrefix(".a*", ".b?", ".c", "-old.d")
## End(Not run)
\end{ExampleCode}
\end{Examples}

\HeaderA{focusWin}{Set the Focus on a Particular Window}{focusWin}
\keyword{methods}{focusWin}
\begin{Description}\relax
Bring the specified window into focus, and set it as the active window. 
\code{focusWin} will fail to bring the window into focus if it is called from the R 
console, since the R console returns focus to itself once a function returns. 
However, it will work if \code{focusWin} is called as a result of calling a function 
from the GUI window. (i.e., pushing a button or any other widget that has a 
function argument).
\end{Description}
\begin{Usage}
\begin{verbatim}focusWin(winName, winVal=TRUE)\end{verbatim}
\end{Usage}
\begin{Arguments}
\begin{ldescription}
\item[\code{winName}] name of window to focus
\item[\code{winVal}] if \code{TRUE}, associate \code{winName} with the default window 
for \code{setWinVal} and \code{getWinVal}
\end{ldescription}
\end{Arguments}
\begin{Author}\relax
Alex Couture-Beil, Malaspina University-College, Nanaimo BC
\end{Author}
\begin{Examples}
\begin{ExampleCode}
## Not run: 
focus <- function() {
  winName <- getWinVal()$select;
  focusWin(winName);
  cat("calling focusWin(\"", winName, "\")\n", sep="");
  cat("getWinVal()$myvar = ", getWinVal()$myvar, "\n\n", sep=""); };

#create three windows named win1, win2, win3
#each having three radio buttons, which are used to change the focus
for(i in 1:3) {
  winDesc <- c(
    paste('window name=win',i,' title="Win',i,'"', sep=''),
    paste('entry myvar ', i, sep=''),
    'radio name=select value=win1 text="one" function=focus mode=character',
    'radio name=select value=win2 text="two" function=focus mode=character',
    'radio name=select value=win3 text="three" function=focus mode=character');
  createWin(winDesc, astext=TRUE); };
## End(Not run)
\end{ExampleCode}
\end{Examples}

\HeaderA{genMatrix}{Generate Test Matrices for plotBubbles}{genMatrix}
\keyword{array}{genMatrix}
\begin{Description}\relax
Generate a test matrix of random numbers (\code{mu} = mean 
and \code{signa} = standard deviation), primarily for \code{plotBubbles}.
\end{Description}
\begin{Usage}
\begin{verbatim}
genMatrix(m,n,mu=0,sigma=1)
\end{verbatim}
\end{Usage}
\begin{Arguments}
\begin{ldescription}
\item[\code{m}] number of rows
\item[\code{n}] number of columns
\item[\code{mu}] mean of normal distribution
\item[\code{sigma}] standard deviation of normal distribution
\end{ldescription}
\end{Arguments}
\begin{Value}
An \code{m} by \code{n} matrix with normally distributed random values.
\end{Value}
\begin{Author}\relax
Jon Schnute, Pacific Biological Station, Nanaimo BC
\end{Author}
\begin{SeeAlso}\relax
\code{\LinkA{plotBubbles}{plotBubbles}}
\end{SeeAlso}
\begin{Examples}
\begin{ExampleCode} plotBubbles(genMatrix(20,6)) \end{ExampleCode}
\end{Examples}

\HeaderA{getChoice}{Choose One String Item from a List of Choices}{getChoice}
\keyword{device}{getChoice}
\keyword{utilities}{getChoice}
\begin{Description}\relax
Prompts the user to choose one string item from a list of 
choices displayed in a GUI. The simplest case \code{getChoice()} 
yields \code{TRUE} or \code{FALSE}.
\end{Description}
\begin{Usage}
\begin{verbatim}
getChoice(choice=c("Yes","No"), question="Make a choice: ",
          winname="getChoice", horizontal=TRUE, radio=FALSE,
          qcolor="blue", gui=FALSE, quiet=FALSE)
\end{verbatim}
\end{Usage}
\begin{Arguments}
\begin{ldescription}
\item[\code{choice }] vector of strings from which to choose.
\item[\code{question }] question or prompting statement.
\item[\code{winname }] window name for the \code{getChoice} GUI.
\item[\code{horizontal }] logical: if \code{TRUE}, display the choices horizontally, else vertically.
\item[\code{radio }] logical: if \code{TRUE}, display the choices as radio buttons, else as buttons.
\item[\code{qcolor }] colour for \code{question}.
\item[\code{gui }] logical: if \code{TRUE}, \code{getChoice} is functional when called from a GUI,
else it is functional from command line programs.
\item[\code{quiet }] logical: if \code{TRUE}, don't print the choice on the command line.
\end{ldescription}
\end{Arguments}
\begin{Details}\relax
The user's choice is stored in \code{.PBSmod\$options\$getChoice} 
(or whatever \code{winname} is supplied).

\code{getChoice} generates an \code{onClose} function that returns focus
to the calling window (if applicable) and prints out the choice.
\end{Details}
\begin{Value}
If called from a GUI (\code{gui=TRUE}), no value is returned directly. Rather, 
the choice is written to the PBS options workspace, accessible through 
\code{getPBSoptions("getChoice")} (or whatever \code{winname} was supplied).

If called from a command line program (\code{gui=FASLE}), the choice is returned 
directly as a string scalar (e.g., \code{answer <- getChoice(gui=F)} ).
\end{Value}
\begin{Note}\relax
Microsoft Windows users may experience difficulties switching focus between the 
R console and GUI windows. The latter frequently disappear from the screen and 
need to be reselected (either clicking on the task bar or pressing \textless{}Alt\textgreater{}\textless{}Tab\textgreater{}. 
This issue can be resolved by switching from MDI to SDI mode. From the R console 
menu bar, select \textless{}Edit\textgreater{} and \textless{}GUI preferences\textgreater{}, then change the value of 
\dQuote{single or multiple windows} to SDI.
\end{Note}
\begin{Author}\relax
Rowan Haigh, Pacific Biological Station, Nanaimo BC
\end{Author}
\begin{SeeAlso}\relax
\code{\LinkA{chooseWinVal}{chooseWinVal}}, \code{\LinkA{getWinVal}{getWinVal}}, \code{\LinkA{setWinVal}{setWinVal}}
\end{SeeAlso}
\begin{Examples}
\begin{ExampleCode}
## Not run: 
#-- Example 1
getChoice(c("Fame","Fortune","Health","Beauty","Lunch"),
   "What do you want?",qcolor="red",gui=F)

#-- Example 2
getChoice(c("Homer Simpson","Wilberforce Humphries","Miss Marple"),
   "Who`s your idol?",horiz=F,radio=T,gui=F)
## End(Not run)
\end{ExampleCode}
\end{Examples}

\HeaderA{getGUIoptions}{Get PBS Options for Widgets}{getGUIoptions}
\begin{Description}\relax
Get the PBS options declared for GUI usage and set their 
corresponding widget values.
\end{Description}
\begin{Usage}
\begin{verbatim}
getGUIoptions()
\end{verbatim}
\end{Usage}
\begin{Details}\relax
The options declared using \code{declareGUIoptions} are 
copied from the R environment into widget values. These widgets 
should have names that match the names of their corresponding options.
\end{Details}
\begin{Author}\relax
Anisa Egeli, Vancouver Island University, Nanaimo BC
\end{Author}
\begin{SeeAlso}\relax
\code{\LinkA{declareGUIoptions}{declareGUIoptions}}, \code{\LinkA{setGUIoptions}{setGUIoptions}},
\code{\LinkA{promptWriteOptions}{promptWriteOptions}}, \code{\LinkA{readPBSoptions}{readPBSoptions}}
\end{SeeAlso}
\begin{Examples}
\begin{ExampleCode}
## Not run: 
getPBSoptions() #loads from default PBSoptions.txt
## End(Not run)
\end{ExampleCode}
\end{Examples}

\HeaderA{getPBSext}{Get a Command Associated With a File Name}{getPBSext}
\keyword{methods}{getPBSext}
\begin{Description}\relax
Display all locally defined file extensions and their associated commands, 
or search for the command associated with a specific file extension
\code{ext}.
\end{Description}
\begin{Usage}
\begin{verbatim}getPBSext(ext)\end{verbatim}
\end{Usage}
\begin{Arguments}
\begin{ldescription}
\item[\code{ext}] optional string specifying a file extension.
\end{ldescription}
\end{Arguments}
\begin{Value}
Command associated with file extension.
\end{Value}
\begin{Note}\relax
These file associations are not saved from one \emph{PBS Modelling} session to
the next unless explicitly saved and loaded (see \code{writePBSoptions} and
\code{readPBSoptions}).
\end{Note}
\begin{Author}\relax
Alex Couture-Beil, Malaspina University-College, Nanaimo BC
\end{Author}
\begin{SeeAlso}\relax
\code{\LinkA{setPBSext}{setPBSext}}, \code{\LinkA{openFile}{openFile}}, \code{\LinkA{clearPBSext}{clearPBSext}}
\end{SeeAlso}

\HeaderA{getPBSoptions}{Retreive A User Option}{getPBSoptions}
\keyword{methods}{getPBSoptions}
\begin{Description}\relax
Get a previously defined user option.
\end{Description}
\begin{Usage}
\begin{verbatim}
getPBSoptions(option)
\end{verbatim}
\end{Usage}
\begin{Arguments}
\begin{ldescription}
\item[\code{option}] name of option to retrieve. If omitted, a list containing all options is returned.
\end{ldescription}
\end{Arguments}
\begin{Value}
Value of the specified option, or \code{NULL} if the specified option is not found.
\end{Value}
\begin{Author}\relax
Alex Couture-Beil, Malaspina University-College, Nanaimo BC
\end{Author}
\begin{SeeAlso}\relax
\code{\LinkA{getPBSext}{getPBSext}}, \code{\LinkA{readPBSoptions}{readPBSoptions}}
\end{SeeAlso}

\HeaderA{getPrefix}{Get Prefix of System Files with Specified Suffix}{getPrefix}
\keyword{character}{getPrefix}
\begin{Description}\relax
Search for and return all string prefixes of system files 
with the specified suffix and system path.
\end{Description}
\begin{Usage}
\begin{verbatim}
getPrefix(suffix, path=".")
\end{verbatim}
\end{Usage}
\begin{Arguments}
\begin{ldescription}
\item[\code{suffix}] string value of suffix (e.g., \code{".txt"}. 
\item[\code{path}] string specifying system path location in which to search. 
\end{ldescription}
\end{Arguments}
\begin{Value}
Vector of string prefixes that have the specified suffix.
\end{Value}
\begin{Author}\relax
Rowan Haigh, Pacific Biological Station, Nanaimo BC
\end{Author}
\begin{SeeAlso}\relax
\code{\LinkA{getSuffix}{getSuffix}}, \code{\LinkA{findPrefix}{findPrefix}}
\end{SeeAlso}

\HeaderA{getSuffix}{Get Suffix of System Files with Specified Prefix}{getSuffix}
\keyword{character}{getSuffix}
\begin{Description}\relax
Search for and return all string suffixes of system files 
with the specified prefix and system path.
\end{Description}
\begin{Usage}
\begin{verbatim}
getSuffix(prefix, path=".")
\end{verbatim}
\end{Usage}
\begin{Arguments}
\begin{ldescription}
\item[\code{prefix}] string value of prefix (e.g., \code{"temp"}. 
\item[\code{path}] string specifying system path location in which to search. 
\end{ldescription}
\end{Arguments}
\begin{Value}
Vector of string suffixes that have the specified prefix.
\end{Value}
\begin{Author}\relax
Rowan Haigh, Pacific Biological Station, Nanaimo BC
\end{Author}
\begin{SeeAlso}\relax
\code{\LinkA{getPrefix}{getPrefix}}, \code{\LinkA{findPrefix}{findPrefix}}
\end{SeeAlso}

\HeaderA{getWinAct}{Retreive the Last Window Action}{getWinAct}
\keyword{methods}{getWinAct}
\begin{Description}\relax
Get a string vector of actions (latest to earliest).
\end{Description}
\begin{Usage}
\begin{verbatim}getWinAct(winName)\end{verbatim}
\end{Usage}
\begin{Arguments}
\begin{ldescription}
\item[\code{winName}] name of window to retrieve action from
\end{ldescription}
\end{Arguments}
\begin{Details}\relax
When a function is called from a GUI, a string descriptor associated with 
the action of the function is stored internally (appended to the first position 
of the action vector). A user can utilize this action as a type of argument 
for programming purposes. The command \code{getWinAct()[1]} yields the latest action.
\end{Details}
\begin{Value}
String vector of recorded actions (latest first).
\end{Value}
\begin{Author}\relax
Alex Couture-Beil, Malaspina University-College, Nanaimo BC
\end{Author}

\HeaderA{getWinFun}{Retrieve Names of Functions Referenced in a Window}{getWinFun}
\keyword{methods}{getWinFun}
\begin{Description}\relax
Get a vector of all function names referenced by a window.
\end{Description}
\begin{Usage}
\begin{verbatim}
getWinFun(winName)
\end{verbatim}
\end{Usage}
\begin{Arguments}
\begin{ldescription}
\item[\code{winName}] name of window, to retrieve its function list
\end{ldescription}
\end{Arguments}
\begin{Value}
A vector of function names referenced by a window.
\end{Value}
\begin{Author}\relax
Alex Couture-Beil, Malaspina University-College, Nanaimo BC
\end{Author}

\HeaderA{getWinVal}{Retreive Widget Values for Use in R Code}{getWinVal}
\keyword{methods}{getWinVal}
\begin{Description}\relax
Get a list of variables defined and set by the GUI widgets. An optional 
argument \code{scope} directs the function to create local or global 
variables based on the list that is returned.
\end{Description}
\begin{Usage}
\begin{verbatim}
getWinVal(v=NULL, scope="", asvector=FALSE, winName="")
\end{verbatim}
\end{Usage}
\begin{Arguments}
\begin{ldescription}
\item[\code{v}] vector of variable names to retrieve from the GUI widgets. 
If \code{NULL}, \code{v} retrieves all variables from all GUI widgets.
\item[\code{scope}] scope of the retrieval. The default sets no variables in the non-GUI 
environment; \code{scope="L"} creates variables locally in relation to the 
parent frame that called the function; and \code{scope="G"} creates global variables(\code{pos=1}).
\item[\code{asvector}] return a vector instead of a list. 
WARNING: if a widget variable defines a true vector or matrix, this will not work.
\item[\code{winName}] window from which to select GUI widget values. The default 
takes the window that has most recently received new user input.
\end{ldescription}
\end{Arguments}
\begin{Details}\relax
TODO talk about scope=G/L and side effects of overwriting existing variables
\end{Details}
\begin{Value}
A list (or vector) with named components, where names and values are defined by GUI widgets.
\end{Value}
\begin{Author}\relax
Alex Couture-Beil, Malaspina University-College, Nanaimo BC
\end{Author}
\begin{SeeAlso}\relax
\code{\LinkA{parseWinFile}{parseWinFile}}, \code{\LinkA{setWinVal}{setWinVal}}, \code{\LinkA{clearWinVal}{clearWinVal}}
\end{SeeAlso}

\HeaderA{getYes}{Prompt the User to Choose Yes or No}{getYes}
\begin{Description}\relax
Display a message prompt with "Yes" and "No" buttons.
\end{Description}
\begin{Usage}
\begin{verbatim}
getYes(message, title="Choice", icon="question")
\end{verbatim}
\end{Usage}
\begin{Arguments}
\begin{ldescription}
\item[\code{message}] message to display in prompt window.
\item[\code{title}] title of prompt window.
\item[\code{icon}] icon to display in prompt window; options are 
\code{"error"}, \code{"info"}, \code{"question"}, or \code{"warning"}.
\end{ldescription}
\end{Arguments}
\begin{Value}
Returns \code{TRUE} if the "Yes" button is clicked, \code{FALSE} if the "No" button is clicked.
\end{Value}
\begin{Author}\relax
Anisa Egeli, Vancouver Island University, Nanaimo BC
\end{Author}
\begin{SeeAlso}\relax
\code{\LinkA{showAlert}{showAlert}}, \code{\LinkA{getChoice}{getChoice}}, \code{\LinkA{chooseWinVal}{chooseWinVal}}
\end{SeeAlso}
\begin{Examples}
\begin{ExampleCode}
## Not run: 
#default settings
if(getYes("Print the number 1?"))
        print(1)
## End(Not run)
\end{ExampleCode}
\end{Examples}

\HeaderA{GT0}{Restrict a Numeric Variable to a Positive Value}{GT0}
\keyword{optimize}{GT0}
\begin{Description}\relax
Restrict a numeric value \code{x} to a positive value using a differentiable function. 
GT0 stands for \dQuote{greater than zero}.
\end{Description}
\begin{Usage}
\begin{verbatim}GT0(x,eps=1e-4)\end{verbatim}
\end{Usage}
\begin{Arguments}
\begin{ldescription}
\item[\code{x}] vector of values
\item[\code{eps}] minimum value greater than zero.
\end{ldescription}
\end{Arguments}
\begin{Details}\relax
\begin{alltt}
   if (x >= eps)..........GT0 = x
   if (0 < x < eps).......GT0 = (eps/2) * (1 + (x/eps)\textasciicircum{}2)
   if (x <= 0)............GT0 = eps/2
\end{alltt}
\end{Details}
\begin{Author}\relax
Jon Schnute, Pacific Biological Station, Nanaimo BC
\end{Author}
\begin{SeeAlso}\relax
\code{\LinkA{scalePar}{scalePar}}, \code{\LinkA{restorePar}{restorePar}}, \code{\LinkA{calcMin}{calcMin}}
\end{SeeAlso}
\begin{Examples}
\begin{ExampleCode}
plotGT0 <- function(eps=1,x1=-2,x2=10,n=1000,col="black") {
        x <- seq(x1,x2,len=n); y <- GT0(x,eps);
        lines(x,y,col=col,lwd=2); invisible(list(x=x,y=y)); }

testGT0 <- function(eps=c(7,5,3,1,.1),x1=-2,x2=10,n=1000) {
        x <- seq(x1,x2,len=n); y <- x;
        plot(x,y,type="l");
        mycol <- c("red","blue","green","brown","violet","orange","pink");
        for (i in 1:length(eps)) 
                plotGT0(eps=eps[i],x1=x1,x2=x2,n=n,col=mycol[i]);
        invisible(); };

testGT0()
\end{ExampleCode}
\end{Examples}

\HeaderA{importHistory}{Import a History List from a File}{importHistory}
\keyword{list}{importHistory}
\begin{Description}\relax
Import a history list from file \code{fname}, and place it 
into the history list \code{hisname}.
\end{Description}
\begin{Usage}
\begin{verbatim}
importHistory(hisname="", fname="", updateHis=TRUE)
\end{verbatim}
\end{Usage}
\begin{Arguments}
\begin{ldescription}
\item[\code{hisname}] name of the history list to be populated. 
The default (\code{""}) uses the value from \code{getWinAct()[1]}.
\item[\code{fname}] file name of history file to import. 
The default (\code{""}) causes an open-file window to be displayed.
\item[\code{updateHis}] logical: if \code{TRUE}, update the history widget to reflect the change
in size and index.
\end{ldescription}
\end{Arguments}
\begin{Author}\relax
Alex Couture-Beil, Malaspina University-College, Nanaimo BC
\end{Author}
\begin{SeeAlso}\relax
\code{\LinkA{exportHistory}{exportHistory}}, \code{\LinkA{initHistory}{initHistory}}, \code{\LinkA{promptOpenFile}{promptOpenFile}}
\end{SeeAlso}

\HeaderA{initHistory}{Create Structures for a New History Widget}{initHistory}
\aliasA{addHistory}{initHistory}{addHistory}
\aliasA{backHistory}{initHistory}{backHistory}
\aliasA{clearHistory}{initHistory}{clearHistory}
\methaliasA{clearHistory}{initHistory}{clearHistory}
\aliasA{firstHistory}{initHistory}{firstHistory}
\aliasA{forwHistory}{initHistory}{forwHistory}
\aliasA{jumpHistory}{initHistory}{jumpHistory}
\aliasA{lastHistory}{initHistory}{lastHistory}
\aliasA{rmHistory}{initHistory}{rmHistory}
\keyword{utilities}{initHistory}
\begin{Description}\relax
PBS history functions (below) are available to those who would like to 
use the package's history functionality, without using the pre-defined history widget. 
These functions allow users to create customized history widgets.
\end{Description}
\begin{Usage}
\begin{verbatim}
initHistory(hisname,indexname=NULL,sizename=NULL,buttonnames=NULL,modename=NULL,
  func=NULL,overwrite=TRUE)
rmHistory(hisname="", index="")
addHistory(hisname="")
forwHistory(hisname="")
backHistory(hisname="")
lastHistory(hisname="")
firstHistory(hisname="")
jumpHistory(hisname="", index="")
clearHistory(hisname="")
\end{verbatim}
\end{Usage}
\begin{Arguments}
\begin{ldescription}
\item[\code{hisname}] name of the history "list" to manipulate. If it is omitted, 
the function uses the value of \code{getWinAct()[1]} as the history name. 
This allows the calling of functions directly from the \emph{window description file} 
(except \code{initHistory}, which must be called before \code{createWin()}).
\item[\code{indexname}] name of the index entry widget in the \emph{window description file}. 
If \code{NULL}, then the current index feature will be disabled.
\item[\code{sizename}] name of the current size entry widget. If \code{NULL}, then the 
current size feature will be disabled.
\item[\code{buttonnames}] named list of names of the first, prev, next, and last buttons. If \code{NULL}, then the 
buttons are not disabled ever
\item[\code{modename}] name of the radio widgets used to change addHistory\'s mode. If \code{NULL}, then the 
default mode will be to insert after the current index.
\item[\code{index}] index to the history item. The default (\code{""}) causes the value to be 
extracted from the widget identified by \code{indexname}.
\item[\code{func}] name of user supplied function to call when viewing history items.
\item[\code{overwrite}] if \code{TRUE}, history (matching \code{hisname}) will be cleared. 
Otherwise, the imported history will be merged with the current one.
\end{ldescription}
\end{Arguments}
\begin{Details}\relax
PBS Modelling includes a pre-built history widget designed to collect interesting choices of
GUI variables so that they can be redisplayed later, rather like a slide show. 

Normally, a user would invoke a history widget simply by including a reference to it 
in the \emph{window description file}. However, PBS Modelling includes support functions (above) 
for customized applications.

To create a customized history, each button must be described separately in the 
\emph{window description file} rather than making reference to the history widget.

The history "List" must be initialized before any other functions may be called. 
The use of a unique history name (\code{hisname}) is used to associate a unique 
history session with the supporting functions.

The \code{indexname} and \code{sizename} arguments correspond to the given names 
of entry widgets in the \emph{window description file}, which will be used to display the 
current index and total size of the list. The \code{indexname} entry widget can also 
be used by \code{jumpHistory} to retrieve a target index.
\end{Details}
\begin{Author}\relax
Alex Couture-Beil, Malaspina University-College, Nanaimo BC
\end{Author}
\begin{SeeAlso}\relax
\code{\LinkA{importHistory}{importHistory}}, \code{\LinkA{exportHistory}{exportHistory}}
\end{SeeAlso}
\begin{Examples}
\begin{ExampleCode}
## Not run: 
# Example of creating a custom history widget that saves values 
# whenever the "Plot" button is pressed. The user can tweak the 
# inputs "a", "b", and "points" before each "Plot" and see the 
# "Index" increase. After sufficient archiving, the user can review 
# scenarios using the "Back" and "Next" buttons. 
# A custom history is needed to achieve this functionality since 
# the packages pre-defined history widget does not update plots.

# To start, create a Window Description to be used with createWin 
# using astext=TRUE. P.S. Watch out for special characters which 
# must be "escaped" twice (first for R, then PBSmodelling).

winDesc <- '
        window title="Custom History"
        vector names="a b k" labels="a b points" font="bold" \\
        values="1 1 1000" function=myPlot
        grid 1 3
                button function=myHistoryBack text="<- Back"
                button function=myPlot text="Plot"
                button function=myHistoryForw text="Next ->"
        grid 2 2
                label "Index"
                entry name="myHistoryIndex" width=5
                label "Size"
                entry name="myHistorySize" width=5
'
# Convert text to vector with each line represented as a new element
winDesc <- strsplit(winDesc, "\n")[[1]]

# Custom functions to update plots after restoring history values
myHistoryBack <- function() {
        backHistory("myHistory");
        myPlot(saveVal=FALSE); # show the plot with saved values
}
myHistoryForw <- function() {
        forwHistory("myHistory");
        myPlot(saveVal=FALSE); # show the plot with saved values 
}
myPlot <- function(saveVal=TRUE) {
        # save all data whenever plot is called (directly)
        if (saveVal) addHistory("myHistory");
        getWinVal(scope="L");
        tt <- 2*pi*(0:k)/k;
        x <- (1+sin(a*tt));  y <- cos(tt)*(1+sin(b*tt));
        plot(x, y);
}

initHistory("myHistory", "myHistoryIndex", "myHistorySize")
createWin(winDesc, astext=TRUE)
## End(Not run)
\end{ExampleCode}
\end{Examples}

\HeaderA{isWhat}{Identify an Object and Print Information}{isWhat}
\keyword{utilities}{isWhat}
\begin{Description}\relax
Identify an object by \code{class}, \code{mode}, \code{typeof},
and \code{attributes}.
\end{Description}
\begin{Usage}
\begin{verbatim}
isWhat(x)
\end{verbatim}
\end{Usage}
\begin{Arguments}
\begin{ldescription}
\item[\code{x}] an R object
\end{ldescription}
\end{Arguments}
\begin{Value}
No value is returned. The function prints the object's 
characteristics on the command line.
\end{Value}
\begin{Author}\relax
Jon Schnute, Pacific Biological Station, Nanaimo BC
\end{Author}

\HeaderA{loadC}{Launch a GUI for Compiling and Loading C Code}{loadC}
\keyword{programming}{loadC}
\keyword{interface}{loadC}
\begin{Description}\relax
A GUI interface allows users to edit, compile, and embed C functions
in the R environment.
\end{Description}
\begin{Usage}
\begin{verbatim}loadC()\end{verbatim}
\end{Usage}
\begin{Details}\relax
The function \code{loadC()} launches an interactive GUI that can be used
to manage the construction of C functions intended to be called from R. The
GUI provides tools to edit, compile, load, and run C functions in the R
environment.

The \code{loadC} GUI also includes a tool for comparison between the running times
and return values of R and C functions. It is assumed that the R and C
functions are named \code{prefix.r} and \code{prefix.c}, respectively, where 
\code{prefix} can be any user-chosen prefix. If an initialization function 
\code{prefix.init} exists, it is called before the start of the comparison.

\bold{The GUI controls:}
\Tabular{ll}{
\bold{File Prefix}   & Prefix for \code{.c} and \code{.r} files.\\
\bold{Lib Prefix}    & Prefix for shared library object.\\
\bold{Set WD}        & Set the working directory.\\
\bold{Open Log}      & Open the log file.\\
\bold{Open.c File}   & Open the file \code{prefix.c} from the working directory.\\
\bold{Open .r File}  & Open the file \code{prefix.r} from the working directory.\\
\bold{COMPILE}       & Compile \code{prefix.c} into a shared library object.\\
\bold{LOAD}          & Load the shared library object.\\
\bold{SOURCE R}      & Source the file \code{prefix.r}.\\
\bold{UNLOAD}        & Unload the shared library object.\\
\bold{Options}       & \\
\bold{Editor}        & Text editor to use.\\
\bold{Update}        & Commit option changes.\\
\bold{Browse}        & Browse for a text editor.\\
\bold{Clean Options} & \\
\bold{Select All}    & Select all check boxes specifying file types.\\
\bold{Select None}   & Select none of the check boxes.\\
\bold{Clean Proj}    & Clean the project of selected file types.\\
\bold{Clean All}     & Clean the directory of selected file types.\\
\bold{Comparison}    & \\
\bold{Times to Run}  & Number of times to run the R and C functions.\\
\bold{RUN}           & Run the comparison between R and C functions.\\
\bold{R Time}        & Computing time to run the R function multiple times.\\
\bold{C Time}        & Computing time to run the C function multiple times.\\
\bold{Ratio}         & Ratio of R/C run times.\\
}
\end{Details}
\begin{Author}\relax
Anisa Egeli, Vancouver Island University, Nanaimo BC
\end{Author}
\begin{SeeAlso}\relax
\code{\LinkA{compileC}{compileC}}
\end{SeeAlso}

\HeaderA{openExamples}{Open Example Files from a Package}{openExamples}
\keyword{file}{openExamples}
\begin{Description}\relax
Open examples from the examples subdirectory of a given package.
\end{Description}
\begin{Usage}
\begin{verbatim}
openExamples(package, prefix, suffix)
\end{verbatim}
\end{Usage}
\begin{Arguments}
\begin{ldescription}
\item[\code{package}] name of the package that contains the examples.
\item[\code{prefix}] prefix of the example file(s).
\item[\code{suffix}] character vector of suffixes for the example files.
\end{ldescription}
\end{Arguments}
\begin{Details}\relax
Copies of each example file are placed in the working directory 
and opened. If files with the same name already exist, the user 
is prompted with a choice to overwrite.

To use this function in a \emph{window description file}, the 
\code{package}, \code{prefix} and \code{suffix} arguments must 
be specified as the action of the widget that calls 
\code{openExamples}. Furthermore, \code{package}, \code{prefix}, 
and each \code{suffix} must be separated by commas. For example, 
\code{action=myPackage,example1,.r,.c} will copy \code{example1.r} 
and \code{example2.c} from the \code{examples} directory of the 
package \pkg{myPackage} to the working directory and open these 
files. If the function was called by a widget, a widget named 
\code{prefix} will be set to the specified prefix.
\end{Details}
\begin{Note}\relax
If all the required arguments are missing, it is assumed 
that the function is being called by a GUI widget.
\end{Note}
\begin{Author}\relax
Anisa Egeli, Vancouver Island University, Nanaimo BC
\end{Author}
\begin{SeeAlso}\relax
\code{\LinkA{openFile}{openFile}}, \code{\LinkA{openProjFiles}{openProjFiles}},
\code{\LinkA{openPackageFile}{openPackageFile}}
\end{SeeAlso}
\begin{Examples}
\begin{ExampleCode}
## Not run: 
# Copies example1.c and example2.r from the examples directory in 
# myPackage to the working directory, and opens these files
openExamples("myPackage", "example1", c(".r", ".c"))
## End(Not run)
\end{ExampleCode}
\end{Examples}

\HeaderA{openFile}{Open a File with an Associated Program}{openFile}
\keyword{file}{openFile}
\begin{Description}\relax
Open a file using the program associated with its extension 
defined by the Windows shell.  Non-windows users, or users 
wishing to overide the default application, can specify a 
program association using \code{setPBSext}.
\end{Description}
\begin{Usage}
\begin{verbatim}
openFile(fname)
\end{verbatim}
\end{Usage}
\begin{Arguments}
\begin{ldescription}
\item[\code{fname}] name of file to open.
\end{ldescription}
\end{Arguments}
\begin{Value}
An invisible string vector of the file names and/or commands + file names.
\end{Value}
\begin{Note}\relax
If a command is registered with \code{setPBSext}, then 
\code{openFile} will replace all occurrences of \code{"\%f"} 
with the absolute path of the filename, before executing the command.
\end{Note}
\begin{Author}\relax
Alex Couture-Beil, Malaspina University-College, Nanaimo BC
\end{Author}
\begin{SeeAlso}\relax
\code{\LinkA{getPBSext}{getPBSext}}, \code{\LinkA{setPBSext}{setPBSext}}, \code{\LinkA{clearPBSext}{clearPBSext}},
\code{\LinkA{writePBSoptions}{writePBSoptions}}
\end{SeeAlso}
\begin{Examples}
\begin{ExampleCode}
## Not run: 
# Set up firefox to open .html files
setPBSext("html", '"c:/Program Files/Mozilla Firefox/firefox.exe" file://%f')
openFile("foo.html")
## End(Not run)
\end{ExampleCode}
\end{Examples}

\HeaderA{openPackageFile}{Open a File from a Package Subdirectory}{openPackageFile}
\keyword{file}{openPackageFile}
\begin{Description}\relax
Open a file from a package in the R library, given the package 
name and the file path relative to the package root directory.
\end{Description}
\begin{Usage}
\begin{verbatim}
openPackageFile(package, filepath)
\end{verbatim}
\end{Usage}
\begin{Arguments}
\begin{ldescription}
\item[\code{package}] name of the package
\item[\code{filepath}] path to file from the package's root directory
\end{ldescription}
\end{Arguments}
\begin{Details}\relax
The \code{openFile} function is used to open the file, using 
associations set by \code{setPBSext}.

To use this function in a \emph{window description file}, the 
\code{package} and \code{filepath} arguments must be specified 
as the action of the widget that calls \code{openPackageFile}. 
Furthermore, \code{package} and \code{filepath} must be 
separated by commas (e.g., \code{action=myPackage,/doc/help.pdf}).
\end{Details}
\begin{Note}\relax
If all the required arguments are missing, it is assumed that 
the function is being called by a GUI widget.
\end{Note}
\begin{Author}\relax
Anisa Egeli, Vancouver Island University, Nanaimo BC
\end{Author}
\begin{SeeAlso}\relax
\code{\LinkA{openFile}{openFile}}, \code{\LinkA{setPBSext}{setPBSext}}, \code{\LinkA{openProjFiles}{openProjFiles}},
\code{\LinkA{openExamples}{openExamples}}
\end{SeeAlso}
\begin{Examples}
\begin{ExampleCode}
## Not run: 
openPackageFile("myPackage", "/doc/help.pdf")
## End(Not run)
\end{ExampleCode}
\end{Examples}

\HeaderA{openProjFiles}{Open Files with a Common Prefix}{openProjFiles}
\keyword{file}{openProjFiles}
\begin{Description}\relax
Open one or more files from the working directory, given one 
file prefix and one or more file suffixes.
\end{Description}
\begin{Usage}
\begin{verbatim}
openProjFiles(prefix, suffix, package=NULL, warn=NULL, alert=TRUE)
\end{verbatim}
\end{Usage}
\begin{Arguments}
\begin{ldescription}
\item[\code{prefix}] a single prefix to prepend to each suffix
\item[\code{suffix}] a character vector of suffixes to append to the prefix
\item[\code{package}] name of the package that contains templates,
or \code{NULL} to not use templates
\item[\code{warn}] if specified, use to temporarily override the 
current R warn option during this function's activity; 
if \code{NULL}, the current warning settings are used.
\item[\code{alert}] if \code{TRUE}, an alert message is shown should 
any files fail to be opened; \\ if \code{FALSE}, no alert is displayed.
\end{ldescription}
\end{Arguments}
\begin{Details}\relax
The suffixes may contain wildcards (\code{"*"} to match 0 or more 
characters, \code{"?"} to match any single character).

For any file that does not exist in the working directory, a template can
optionally be copied from a directory named \code{templates} in the specified
package. The templates in this directory should have the prefix \code{template},
followed by the suffix to match when \code{openProjFiles} is called
(e.g., \code{template.c} to match the suffix \code{.c}. After being copied 
to the working directory, the new file is renamed to use the specified prefix.

To use this function in a \emph{window description file}, the
\code{package} and \code{suffix} arguments must be specified as the action of the widget
that calls \code{openProjFiles}. Furthermore, \code{package} and each \code{suffix} must be
separated by commas. For example, \code{action=myPackage,.r,.c} will try to open a \code{.r}
and \code{.c} file in the working directory, copying templates from the \code{template}
directory for the package \pkg{myPackage}, if the files didn't already exist. To
disable templates, leave \code{package} unspecified but keep the leading comma
(e.g., \code{action=,.r,.c}). When the function is called from a widget in this
fashion, the prefix is taken from the value of a widget named \code{prefix}.
\end{Details}
\begin{Note}\relax
If all the required arguments are missing, it is assumed that the 
function is being called by a GUI widget.
\end{Note}
\begin{Author}\relax
Anisa Egeli, Vancouver Island University, Nanaimo BC
\end{Author}
\begin{SeeAlso}\relax
\code{\LinkA{openFile}{openFile}}, \code{\LinkA{setPBSext}{setPBSext}}, \code{\LinkA{openExamples}{openExamples}},
\code{\LinkA{openPackageFile}{openPackageFile}}
\end{SeeAlso}
\begin{Examples}
\begin{ExampleCode}
## Not run: 
openProjFiles("foo", c(".r", ".c"), package="myPackage")
## End(Not run)
\end{ExampleCode}
\end{Examples}

\HeaderA{packList}{Pack a List with Objects}{packList}
\keyword{list}{packList}
\keyword{file}{packList}
\begin{Description}\relax
Pack a list with existing objects using names only.
\end{Description}
\begin{Usage}
\begin{verbatim}
packList(stuff, target="PBSlist", value, 
         lenv=parent.frame(), tenv=.GlobalEnv)
\end{verbatim}
\end{Usage}
\begin{Arguments}
\begin{ldescription}
\item[\code{stuff}] string vector of object names 
\item[\code{target}] target list object 
\item[\code{value}] an optional explicit value to assign to \code{stuff} 
\item[\code{lenv}] local environment where objects are located 
\item[\code{tenv}] target environment where target list is or will be located 
\end{ldescription}
\end{Arguments}
\begin{Details}\relax
A list object called \code{target} will be located in the 
\code{tenv} environment. The objects named in \code{stuff} and 
located in the \code{lenv} environment will appear as named 
components within the list object \code{target}.

If an explicit \code{value} is specified, the function uses this value
instead of looking for local objects. Essentially, \code{stuff=value}
which is then packed into \code{target}.
\end{Details}
\begin{Value}
No value is returned
\end{Value}
\begin{Author}\relax
Rowan Haigh, Pacific Biological Station, Nanaimo BC
\end{Author}
\begin{SeeAlso}\relax
\code{\LinkA{unpackList}{unpackList}}, \code{\LinkA{readList}{readList}}, \code{\LinkA{writeList}{writeList}}
\end{SeeAlso}
\begin{Examples}
\begin{ExampleCode}
fn = function() {
        alpha=rnorm(10)
        beta=letters
        gamma=mean
        delta=longley
        packList(c("alpha","beta","gamma","delta")) }
fn(); print(PBSlist)
\end{ExampleCode}
\end{Examples}

\HeaderA{pad0}{Pad Numbers with Leading Zeroes}{pad0}
\keyword{print}{pad0}
\begin{Description}\relax
Convert numbers to integers then text, and pad them with leading zeroes.
\end{Description}
\begin{Usage}
\begin{verbatim}
pad0(x, n, f = 0)
\end{verbatim}
\end{Usage}
\begin{Arguments}
\begin{ldescription}
\item[\code{x}] vector of numbers
\item[\code{n}] number of text characters representing a padded integer
\item[\code{f}] factor of 10 transformation on x before padding
\end{ldescription}
\end{Arguments}
\begin{Value}
A character vector representing \code{x} with leading zeroes.
\end{Value}
\begin{Author}\relax
Rowan Haigh, Pacific Biological Station, Nanaimo BC
\end{Author}
\begin{Examples}
\begin{ExampleCode}
resetGraph(); x <- pad0(x=123,n=10,f=0:7);
addLabel(.5,.5,paste(x,collapse="\n"),cex=1.5);
\end{ExampleCode}
\end{Examples}

\HeaderA{parseWinFile}{Convert a Window Description File into a List Object}{parseWinFile}
\keyword{list}{parseWinFile}
\begin{Description}\relax
Parse a \emph{window description file} (markup file) into 
the list format expected by \code{createWin}.
\end{Description}
\begin{Usage}
\begin{verbatim}
parseWinFile(fname, astext=FALSE)
\end{verbatim}
\end{Usage}
\begin{Arguments}
\begin{ldescription}
\item[\code{fname}] file name of the \emph{window description file}.
\item[\code{astext}] if \code{TRUE}, \code{fname} is interpreted as a vector of strings, 
with each element representing a line of code in a \emph{window description file}.
\end{ldescription}
\end{Arguments}
\begin{Value}
A list representing a parsed \emph{window description file} that can be directly 
passed to \code{createWin}.
\end{Value}
\begin{Note}\relax
All widgets are forced into a 1-column by N-row grid.
\end{Note}
\begin{Author}\relax
Alex Couture-Beil, Malaspina University-College, Nanaimo BC
\end{Author}
\begin{SeeAlso}\relax
\code{\LinkA{createWin}{createWin}}, \code{\LinkA{compileDescription}{compileDescription}}
\end{SeeAlso}
\begin{Examples}
\begin{ExampleCode}
## Not run: 
x<-parseWinFile(system.file("examples/LissFigWin.txt",package="PBSmodelling"))
createWin(x)
## End(Not run)
\end{ExampleCode}
\end{Examples}

\HeaderA{pause}{Pause Between Graphics Displays or Other Calculations}{pause}
\keyword{utilities}{pause}
\begin{Description}\relax
Pause, typically between graphics displays. Useful for demo purposes.
\end{Description}
\begin{Usage}
\begin{verbatim}
pause(s = "Press <Enter> to continue")
\end{verbatim}
\end{Usage}
\begin{Arguments}
\begin{ldescription}
\item[\code{s}] text issued on the command line when \code{pause} is invoked.
\end{ldescription}
\end{Arguments}
\begin{Author}\relax
Jon Schnute, Pacific Biological Station, Nanaimo BC
\end{Author}

\HeaderA{PBSmodelling}{PBS Modelling}{PBSmodelling}
\aliasA{PBSmodelling-package}{PBSmodelling}{PBSmodelling.Rdash.package}
\keyword{package}{PBSmodelling}
\begin{Description}\relax
\emph{PBS Modelling} provides software to facilitate the design, 
testing, and operation of computer models. It focuses particularly on 
tools that make it easy to construct and edit a customized graphical 
user interface (GUI). Although it depends heavily on the R interface 
to the \code{Tcl/Tk} package, a user does not need to know Tcl/Tk. 

\code{PBSmodelling} contains examples that illustrate models built using 
other R packages, including \code{PBSmapping}, \code{odesolve}, 
\code{PBSddesolve}, and \code{BRugs}. 
It also serves as a convenient prototype for building new R packages, 
along with instructions and batch files to facilitate that process.

The R directory \code{.../library/PBSmodelling/doc} includes a complete 
user guide \sQuote{PBSmodelling-UG.pdf}. To use this package effectively, 
please consult the guide.

\emph{PBS Modelling} comes packaged with interesting examples accessed 
through the function \code{runExamples()}.
Additionally, users can view \emph{PBS Modelling} widgets through the 
function \code{testWidgets()}. 
More generally, a user can run any available demos in his/her locally 
installed packages through the function \code{runDemos()}.
\end{Description}

\HeaderA{pickCol}{Pick a Colour From a Palette and get the Hexadecimal Code}{pickCol}
\keyword{color}{pickCol}
\begin{Description}\relax
Display an interactive colour palette from which the user can choose a colour.
\end{Description}
\begin{Usage}
\begin{verbatim}pickCol(returnValue=TRUE)\end{verbatim}
\end{Usage}
\begin{Arguments}
\begin{ldescription}
\item[\code{returnValue}] If \code{TRUE}, display the full colour palette, choose a colour, 
and return the hex value to the R session. \\ If \code{FALSE}, use an intermediate 
GUI to interact with the palette and display the hex value of the chosen colour.
\end{ldescription}
\end{Arguments}
\begin{Value}
A hexidecimal colour value.
\end{Value}
\begin{Author}\relax
Alex Couture-Beil, Malaspina University-College, Nanaimo BC
\end{Author}
\begin{SeeAlso}\relax
\code{\LinkA{testCol}{testCol}}
\end{SeeAlso}
\begin{Examples}
\begin{ExampleCode}
## Not run: 
junk<-pickCol(); resetGraph(); addLabel(.5,.5,junk,cex=4,col=junk);
## End(Not run)
\end{ExampleCode}
\end{Examples}

\HeaderA{plotACF}{Plot Autocorrelation Bars From a Data Frame, Matrix, or Vector}{plotACF}
\keyword{graphs}{plotACF}
\begin{Description}\relax
Plot autocorrelation bars (ACF) from a data frame, matrix, or vector.
\end{Description}
\begin{Usage}
\begin{verbatim}
plotACF(file, lags=20, 
        clrs=c("blue","red","green","magenta","navy"), ...)
\end{verbatim}
\end{Usage}
\begin{Arguments}
\begin{ldescription}
\item[\code{file}] data frame, matrix, or vector of numeric values.
\item[\code{lags}] maximum number of lags to use in the ACF calculation.
\item[\code{clrs}] vector of colours. Patterns are repeated if the number 
of fields exceed the length of \code{clrs}.
\item[\code{...}] additional arguments for \code{plot} or \code{lines}.
\end{ldescription}
\end{Arguments}
\begin{Details}\relax
This function is designed primarily to give greater flexibility when viewing 
results from the R-package \code{BRugs}. Use \code{plotACF} in conjunction with 
\code{samplesHistory("*",beg=0,plot=FALSE)} rather than \code{samplesAutoC} 
which calls \code{plotAutoC}.
\end{Details}
\begin{Author}\relax
Rowan Haigh, Pacific Biological Station, Nanaimo BC
\end{Author}
\begin{Examples}
\begin{ExampleCode}
resetGraph(); plotACF(trees,lwd=2,lags=30);
\end{ExampleCode}
\end{Examples}

\HeaderA{plotAsp}{Construct a Plot with a Specified Aspect Ratio}{plotAsp}
\keyword{hplot}{plotAsp}
\begin{Description}\relax
Plot \code{x} and \code{y} coordinates using a specified aspect ratio.
\end{Description}
\begin{Usage}
\begin{verbatim}
plotAsp(x, y, asp=1, ...)
\end{verbatim}
\end{Usage}
\begin{Arguments}
\begin{ldescription}
\item[\code{x}] vector of x-coordinate points in the plot.
\item[\code{y}] vector of y-coordinate points in the plot.
\item[\code{asp}] \code{y}/\code{x} aspect ratio.
\item[\code{...}] additional arguments for \code{plot}.
\end{ldescription}
\end{Arguments}
\begin{Details}\relax
The function \code{plotAsp} differs from \code{plot(x,y,asp=1)} in the way axis 
limits are handled. Rather than expand the range, \code{plotAsp} expands the 
margins through padding to keep the aspect ratio accurate.
\end{Details}
\begin{Author}\relax
Alex Couture-Beil, Malaspina University-College, Nanaimo BC
\end{Author}
\begin{Examples}
\begin{ExampleCode}
x <- seq(0,10,0.1)
y <- sin(x)
par(mfrow=2:1)
plotAsp(x,y,asp=1,xlim=c(0,10),ylim=c(-2,2), main="sin(x)")
plotAsp(x,y^2,asp=1,xlim=c(0,10),ylim=c(-2,2), main="sin^2(x)")
\end{ExampleCode}
\end{Examples}

\HeaderA{plotBubbles}{Construct a Bubble Plot from a Matrix}{plotBubbles}
\keyword{hplot}{plotBubbles}
\begin{Description}\relax
Construct a bubble plot for a matrix \code{z}.
\end{Description}
\begin{Usage}
\begin{verbatim}
plotBubbles(z, xval=FALSE, yval=FALSE, dnam=FALSE, rpro=FALSE, 
   cpro=FALSE, rres=FALSE, cres=FALSE, powr=0.5, size=0.2, lwd=1,
   clrs=c("black","red","blue"), hide0=FALSE, frange=0.1, ...) 
\end{verbatim}
\end{Usage}
\begin{Arguments}
\begin{ldescription}
\item[\code{z}] input matrix, array (2 dimensions) or data frame.
\item[\code{xval}] x-values and/or labels for the columns of \code{z}. 
if \code{xval=TRUE}, the first row contains x-values for the columns.
\item[\code{yval}] y-values and/or labels for the rows of \code{z}. 
If \code{yval=TRUE}, the first column contains y-values for the rows.
\item[\code{dnam}] logical: if \code{TRUE}, attempt to use \code{dimnames} of input
matrix \code{z} as \code{xval} and \code{yval}. The \code{dimnames} are 
converted to numeric values and must be strictly increasing or decreasing. 
If successful, these values will overwrite previously specified values
of \code{xval} and \code{yval} or any default indices.
\item[\code{rpro}] logical: if \code{TRUE}, convert rows to proportions.
\item[\code{cpro}] logical: if \code{TRUE}, convert columns to proportions.
\item[\code{rres}] logical: if \code{TRUE}, use row residuals (subtract row means).
\item[\code{cres}] logical: if \code{TRUE}, use column residuals (subtract column means).
\item[\code{powr}] power transform. Radii are proportional to \code{z\textasciicircum{}powr}. 
Note: \code{powr=0.5} yields bubble areas proportional to \code{z}.
\item[\code{size}] size (inches) of the largest bubble.
\item[\code{lwd}] line width for drawing circles.
\item[\code{clrs}] colours (3-element vector) used for positive, negative, 
and zero values, respectively.
\item[\code{hide0}] logical: if \code{TRUE}, hide zero-value bubbles.
\item[\code{frange}] number specifying the fraction by which the range of the axes should be extended.
\item[\code{...}] additional arguments for plotting functions.
\end{ldescription}
\end{Arguments}
\begin{Details}\relax
The function \code{plotBubbles} essentially flips the \code{z} matrix 
visually. The columns of \code{z} become the x-values while the rows of
\code{z} become the y-values, where the first row is displayed as the
bottom y-value and the last row is displayed as the top y-value. The 
function's original intention was to display proportions-at-age vs. year.
\end{Details}
\begin{Author}\relax
Jon Schnute, Pacific Biological Station, Nanaimo BC
\end{Author}
\begin{SeeAlso}\relax
\code{\LinkA{genMatrix}{genMatrix}}
\end{SeeAlso}
\begin{Examples}
\begin{ExampleCode}
plotBubbles(round(genMatrix(40,20),0),clrs=c("green","grey","red"));

data(CCA.qbr)
plotBubbles(CCA.qbr,cpro=TRUE,powr=.5,dnam=TRUE,size=.15,
   ylim=c(0,70),xlab="Year",ylab="Quillback Rockfish Age")
\end{ExampleCode}
\end{Examples}

\HeaderA{plotCsum}{Plot Cumulative Sum of Data}{plotCsum}
\keyword{hplot}{plotCsum}
\begin{Description}\relax
Plot the cumulative frequency of a data vector or matrix, 
showing the median and mean of the distribution.
\end{Description}
\begin{Usage}
\begin{verbatim}
plotCsum(x, add = FALSE, ylim = c(0, 1), xlab = "Measure", 
ylab = "Cumulative Proportion", ...)  
\end{verbatim}
\end{Usage}
\begin{Arguments}
\begin{ldescription}
\item[\code{x}] vector or matrix of numeric values.
\item[\code{add}] logical: if \code{TRUE}, add the cumulative frequency curve to a current plot.
\item[\code{ylim}] limits for the y-axis.
\item[\code{xlab}] label for the x-axis.
\item[\code{ylab}] label for the y-axis.
\item[\code{...}] additional arguments for the \code{plot} function.
\end{ldescription}
\end{Arguments}
\begin{Author}\relax
Rowan Haigh, Pacific Biological Station, Nanaimo BC
\end{Author}
\begin{Examples}
\begin{ExampleCode}
x <- rgamma(n=1000,shape=2)
plotCsum(x)
\end{ExampleCode}
\end{Examples}

\HeaderA{plotDens}{Plot Density Curves from a Data Frame, Matrix, or Vector}{plotDens}
\keyword{graphs}{plotDens}
\begin{Description}\relax
Plot the density curves from a data frame, matrix, or vector. 
The mean density curve of the data combined is also shown.
\end{Description}
\begin{Usage}
\begin{verbatim}
plotDens(file, clrs=c("blue","red","green","magenta","navy"), ...)  
\end{verbatim}
\end{Usage}
\begin{Arguments}
\begin{ldescription}
\item[\code{file}] data frame, matrix, or vector of numeric values.
\item[\code{clrs}] vector of colours. Patterns are repeated if the number 
of fields exceed the length of \code{clrs}.
\item[\code{...}] additional arguments for \code{plot} or \code{lines}.
\end{ldescription}
\end{Arguments}
\begin{Details}\relax
This function is designed primarily to give greater flexibility when viewing 
results from the R-package \code{BRugs}. Use \code{plotDens} in conjunction with 
\code{samplesHistory("*",beg=0,plot=FALSE)} rather than \code{samplesDensity} 
which calls \code{plotDensity}.
\end{Details}
\begin{Author}\relax
Rowan Haigh, Pacific Biological Station, Nanaimo BC
\end{Author}
\begin{Examples}
\begin{ExampleCode}
z <- data.frame(y1=rnorm(50,sd=2),y2=rnorm(50,sd=1),y3=rnorm(50,sd=.5))
plotDens(z,lwd=3)
\end{ExampleCode}
\end{Examples}

\HeaderA{plotFriedEggs}{Render a Pairs Plot as Fried Eggs and Beer}{plotFriedEggs}
\keyword{hplot}{plotFriedEggs}
\begin{Description}\relax
Create a pairs plot where the lower left half comprises either 
fried egg contours or smoke ring contours, the upper right half 
comprises glasses of beer filled to the correlation point, and 
the diagonals show frequency histograms of the input data.
\end{Description}
\begin{Usage}
\begin{verbatim}
plotFriedEggs(A, eggs=TRUE, rings=TRUE, levs=c(0.01,0.1,0.5,0.75,0.95),
              pepper=200, replace=FALSE, jitt=c(1,1), bw=25, histclr=NULL) 
\end{verbatim}
\end{Usage}
\begin{Arguments}
\begin{ldescription}
\item[\code{A}] data frame or matrix for use in a pairs plot.
\item[\code{eggs}] logical: if \code{TRUE}, fry eggs in the lower panels.
\item[\code{rings}] logical: if \code{TRUE}, blow smoke rings in the lower panels.
\item[\code{levs}] explicit contour levels expressed as quantiles.
\item[\code{pepper}] number of samples to draw from \code{A} to pepper the plots.
\item[\code{replace}] logical: if \code{TRUE}, sample \code{A} with replacement.
\item[\code{jitt}] argument \code{factor} used by function \code{base::jitter} when peppering.
If user supplies two numbers, the first will jitter \code{x}, the second will 
jitter \code{y}.
\item[\code{bw}] argument \code{bandwidth} used by function \code{KernSmooth::bkde2D}.
\item[\code{histclr}] user-specified colour(s) for histogram bars along the diagonal.
\end{ldescription}
\end{Arguments}
\begin{Details}\relax
This function comes to us from Dr. Steve Martell of the Fisheries Science 
Centre at UBC. Obviously many hours of contemplation with his students at
the local pub have contributed to this unique rendition of a pairs plot.
\end{Details}
\begin{Note}\relax
If \code{eggs=TRUE} and \code{rings=FALSE}, fried eggs are served.\\
If \code{eggs=FALSE} and \code{rings=TRUE}, smoke rings are blown.\\
If \code{eggs=TRUE} and \code{rings=TRUE}, only fried eggs are served.\\
If \code{eggs=FALSE} and \code{rings=FALSE}, only pepper is sprinkled.
\end{Note}
\begin{Author}\relax
Steve Martell, University of British Columbia, Vancouver BC
\end{Author}
\begin{SeeAlso}\relax
\code{\LinkA{plotBubbles}{plotBubbles}}, \code{\LinkA{scalePar}{scalePar}}

\code{KernSmooth::bkde2D}, \code{grDevices::contourLines}, \code{graphics::contour}
\end{SeeAlso}
\begin{Examples}
\begin{ExampleCode}
x=rnorm(5000,10,3); y=-x+rnorm(5000,1,4); z=x+rnorm(5000,1,3)
A=data.frame(x=x,y=y,z=z)
for (i in 1:3)
   switch(i,
   {plotFriedEggs(A,eggs=TRUE,rings=FALSE);
   pause("Here are the eggs...(Press Enter for next)")},
   {plotFriedEggs(A,eggs=FALSE,rings=TRUE);
   pause("Here are the rings...(Press Enter for next)")},
   {plotFriedEggs(A,eggs=FALSE,rings=FALSE);
   cat("Here is the pepper alone.\n")} )
\end{ExampleCode}
\end{Examples}

\HeaderA{plotTrace}{Plot Trace Lines from a Data Frame, Matrix, or Vector}{plotTrace}
\keyword{graphs}{plotTrace}
\begin{Description}\relax
Plot trace lines from a data frame or matrix where the first field contains x-values, 
and subsequent fields give y-values to be traced over x. If input is a vector, 
this is traced over the number of observations.
\end{Description}
\begin{Usage}
\begin{verbatim}
plotTrace(file, clrs=c("blue","red","green","magenta","navy"), ...)
\end{verbatim}
\end{Usage}
\begin{Arguments}
\begin{ldescription}
\item[\code{file}] data frame or matrix of x and y-values, or a vector of y-values.
\item[\code{clrs}] vector of colours. Patterns are repeated if the number 
of traces (y-fields) exceed the length of \code{clrs}.
\item[\code{...}] additional arguments for \code{plot} or \code{lines}.
\end{ldescription}
\end{Arguments}
\begin{Details}\relax
This function is designed primarily to give greater flexibility when viewing 
results from the R-package \code{BRugs}. Use \code{plotTrace} in conjunction with 
\code{samplesHistory("*",beg=0,plot=FALSE)} rather than \code{samplesHistory} 
which calls \code{plotHistory}.
\end{Details}
\begin{Author}\relax
Rowan Haigh, Pacific Biological Station, Nanaimo BC
\end{Author}
\begin{Examples}
\begin{ExampleCode}
z <- data.frame(x=1:50,y1=rnorm(50,sd=3),y2=rnorm(50,sd=1),y3=rnorm(50,sd=.25))
plotTrace(z,lwd=3)
\end{ExampleCode}
\end{Examples}

\HeaderA{presentTalk}{Run an R Presentation}{presentTalk}
\begin{Description}\relax
Start an R talk from a \emph{talk description file} that launches a control GUI.
\end{Description}
\begin{Usage}
\begin{verbatim}
presentTalk(x, debug=FALSE)
\end{verbatim}
\end{Usage}
\begin{Arguments}
\begin{ldescription}
\item[\code{x}] string name of \emph{talk description file}.
\item[\code{debug}] logical: if \code{TRUE}, the command line reflects indices and some booleans.
\end{ldescription}
\end{Arguments}
\begin{Details}\relax
\code{presentTalk} is a tool that facilitates lectures and workshops in R.
The function allows the presenter to show code snippets alongside their 
execution, making use of R's graphical capabilities. 
When \code{presentTalk} is called, a graphical user interface (GUI) is 
launched that allows the user to control the flow of the talk (e.g., 
switching between talks or skipping to various sections of a talk.

The automatic control buttons allow the user
to move forward or backward in the talk. The \code{GO} button moves forward 
one tag segment, the \code{Back} button moves back to the previous tag segment.
The blue buttons allow movement among sections - \code{Start} to the first 
section of the talk, \code{Prev} to the previous section, \code{Curr} to the 
start of the current section, and \code{Next} to the next section.

In addition to the automatic menu items, a user can add buttons to the GUI 
that accomplish similar purposes.
\end{Details}
\begin{Note}\relax
The use of \code{chunk} in the R code is equivalent to the use of 
\code{segment} in the documentation.\\
See the PBSmodelling User's Guide for more information.
\end{Note}
\begin{Author}\relax
Anisa Egeli, Vancouver Island University, Nanaimo BC
\end{Author}

\HeaderA{promptOpenFile}{Display Dialogue: Open File}{promptOpenFile}
\keyword{file}{promptOpenFile}
\begin{Description}\relax
Display the default \bold{Open} prompt provided by the Operating System.
\end{Description}
\begin{Usage}
\begin{verbatim}
promptOpenFile(initialfile="", filetype=list(c("*","All Files")), 
               open=TRUE)
\end{verbatim}
\end{Usage}
\begin{Arguments}
\begin{ldescription}
\item[\code{initialfile}] file name of the text file containing the list.
\item[\code{filetype}] a list of character vectors indicating file types made available 
to users of the GUI. Each vector is of length one or two. The first element 
specifies either the file extension or \code{"*"} for all file types. The second 
element gives an optional descriptor name for the file type. The supplied 
\code{filetype} list appears as a set of choices in the pull-down box labelled 
\dQuote{Files of type:"}.
\item[\code{open}] logical: if \code{TRUE} display \bold{Open} prompt, 
if \code{FALSE} display \bold{Save As} prompt.
\end{ldescription}
\end{Arguments}
\begin{Value}
The file name and path of the file selected by the user.
\end{Value}
\begin{Author}\relax
Alex Couture-Beil, Malaspina University-College, Nanaimo BC
\end{Author}
\begin{SeeAlso}\relax
\code{\LinkA{promptSaveFile}{promptSaveFile}}
\end{SeeAlso}
\begin{Examples}
\begin{ExampleCode}
## Not run: 
# Open a filename, and return it line by line in a vector
scan(promptOpenFile(),what=character(),sep="\n")

# Illustrates how to set filetype.
promptOpenFile("intial_file.txt", filetype=list(c(".txt", "text files"), 
               c(".r", "R files"), c("*", "All Files")))
## End(Not run)
\end{ExampleCode}
\end{Examples}

\HeaderA{promptSaveFile}{Display Dialogue: Save File}{promptSaveFile}
\keyword{file}{promptSaveFile}
\begin{Description}\relax
Display the default \bold{Save As} prompt provided by the Operating System.
\end{Description}
\begin{Usage}
\begin{verbatim}
promptSaveFile(initialfile="", filetype=list(c("*", "All Files")), 
               save=TRUE)
\end{verbatim}
\end{Usage}
\begin{Arguments}
\begin{ldescription}
\item[\code{initialfile}] file name of the text file containing the list.
\item[\code{filetype}] a list of character vectors indicating file types made available 
to users of the GUI. Each vector is of length one or two. The first element 
specifies either the file extension or \code{"*"} for all file types. The second 
element gives an optional descriptor name for the file type. The supplied 
\code{filetype} list appears as a set of choices in the pull-down box labelled 
\dQuote{Files of type:}.
\item[\code{save}] logical: if \code{TRUE} display \bold{Save As} prompt, 
if \code{FALSE} display \bold{Open} prompt.
\end{ldescription}
\end{Arguments}
\begin{Value}
The file name and path of the file selected by the user.
\end{Value}
\begin{Author}\relax
Alex Couture-Beil, Malaspina University-College, Nanaimo BC
\end{Author}
\begin{SeeAlso}\relax
\code{\LinkA{promptOpenFile}{promptOpenFile}}
\end{SeeAlso}
\begin{Examples}
\begin{ExampleCode}
## Not run: 
#illustrates how to set filetype.
promptSaveFile("intial_file.txt", filetype=list(c(".txt", "text files"), 
               c(".r", "R files"), c("*", "All Files")))
## End(Not run)
\end{ExampleCode}
\end{Examples}

\HeaderA{promptWriteOptions}{Prompt the User to Write Changed Options}{promptWriteOptions}
\begin{Description}\relax
If changes have been made to PBS options, this function allows 
the user to choose whether to write PBS options to an external 
file that can be loaded later by \code{readPBSoptions}.
\end{Description}
\begin{Usage}
\begin{verbatim}
promptWriteOptions(fname="")
\end{verbatim}
\end{Usage}
\begin{Arguments}
\begin{ldescription}
\item[\code{fname}] name of file where options will be saved.
\end{ldescription}
\end{Arguments}
\begin{Details}\relax
If there are options that have been changed in the GUI but have not been
committed to PBSmodelling memory in the global R environment, the user
is prompted to choose whether or not to commit these options.

Then, if any PBS options have been changed, the user is prompted to choose
whether to save these options to the file \code{fname}. (When a new R session is
started or when a call to \code{readPBSoptions} or \code{writePBSoptions} is made,
PBS options are considered to be unchanged; when an option is set,
the options are considered to be changed).

If \code{fname=""}, the user is prompted to save under the file name last used 
by a call to \code{readPBSoptions} or \code{writePBSoptions} if available. 
Otherwise, the default file name "PBSoptions.txt" is used.
\end{Details}
\begin{Author}\relax
Anisa Egeli, Vancouver Island University, Nanaimo BC
\end{Author}
\begin{SeeAlso}\relax
\code{\LinkA{writePBSoptions}{writePBSoptions}}, \code{\LinkA{readPBSoptions}{readPBSoptions}},
\code{\LinkA{setPBSoptions}{setPBSoptions}}
\end{SeeAlso}
\begin{Examples}
\begin{ExampleCode}
## Not run: 
promptWriteOptions() #uses default filename PBSoptions.txt
## End(Not run)
\end{ExampleCode}
\end{Examples}

\HeaderA{readList}{Read a List from a File in PBS Modelling Format}{readList}
\keyword{list}{readList}
\keyword{file}{readList}
\begin{Description}\relax
Read in a list previously saved to a file by \code{writeList}. 
At present, only two formats are supported - R's native format 
used by the \code{dput} function or an ad hoc \code{PBSmodelling} format. 
The function \code{readList} detects the format automatically.

For information about the \code{PBSmodelling} format, see \code{writeList}.
\end{Description}
\begin{Usage}
\begin{verbatim}
readList(fname)
\end{verbatim}
\end{Usage}
\begin{Arguments}
\begin{ldescription}
\item[\code{fname}] file name of the text file containing the list.
\end{ldescription}
\end{Arguments}
\begin{Author}\relax
Alex Couture-Beil, Malaspina University-College, Nanaimo BC
\end{Author}
\begin{SeeAlso}\relax
\code{\LinkA{packList}{packList}}, \code{\LinkA{unpackList}{unpackList}}, \code{\LinkA{writeList}{writeList}}
\end{SeeAlso}

\HeaderA{readPBSoptions}{Read PBS Options from an External File}{readPBSoptions}
\begin{Description}\relax
Load options that were saved using \code{writePBSoptions}, for use
with \code{openFile}, \code{getPBSoptions} or interfaces such as
\code{loadC}.
\end{Description}
\begin{Usage}
\begin{verbatim}
readPBSoptions(fname="PBSoptions.txt")
\end{verbatim}
\end{Usage}
\begin{Arguments}
\begin{ldescription}
\item[\code{fname}] file name or full path of file from which the options will be loaded.
\end{ldescription}
\end{Arguments}
\begin{Note}\relax
If an option exists in R memory but not in the saved file, 
the option is not cleared from memory.
\end{Note}
\begin{Author}\relax
Anisa Egeli, Vancouver Island University, Nanaimo BC
\end{Author}
\begin{SeeAlso}\relax
\code{\LinkA{writePBSoptions}{writePBSoptions}}, \code{\LinkA{getGUIoptions}{getGUIoptions}},
\code{\LinkA{openFile}{openFile}}, \code{\LinkA{getPBSoptions}{getPBSoptions}}
\end{SeeAlso}

\HeaderA{resetGraph}{Reset par Values for a Plot}{resetGraph}
\keyword{device}{resetGraph}
\begin{Description}\relax
Reset \code{par()} to default values to ensure that a new plot 
utilizes a full figure region. This function helps manage the device 
surface, especially after previous plotting has altered it.
\end{Description}
\begin{Usage}
\begin{verbatim}resetGraph(reset.mf=TRUE)\end{verbatim}
\end{Usage}
\begin{Arguments}
\begin{ldescription}
\item[\code{reset.mf}] if \code{TRUE} reset the multi-frame status; otherwise
preserve \code{mfrow}, \code{mfcol}, and \code{mfg}
\end{ldescription}
\end{Arguments}
\begin{Details}\relax
This function resets \code{par()} to its default values.
If \code{reset.mf=TRUE}, it also clears the graphics device with 
\code{frame()}. Otherwise, the values of \code{mfrow}, \code{mfcol}, 
and \code{mfg} are preserved, and graphics continues as usual in
the current plot. Use \code{resetGraph} only before a high level
command that would routinely advance to a new frame.
\end{Details}
\begin{Value}
invisible return of the reset value \code{par()}
\end{Value}
\begin{Author}\relax
Jon Schnute, Pacific Biological Station, Nanaimo BC
\end{Author}

\HeaderA{restorePar}{Get Actual Parameters from Scaled Values}{restorePar}
\keyword{optimize}{restorePar}
\begin{Description}\relax
Restore scaled parameters to their original units. Used in minimization by \code{calcMin}.
\end{Description}
\begin{Usage}
\begin{verbatim}restorePar(S,pvec)\end{verbatim}
\end{Usage}
\begin{Arguments}
\begin{ldescription}
\item[\code{S}] scaled parameter vector.
\item[\code{pvec}] a data frame comprising four columns - 
\code{c("val","min","max","active")} and as many rows as there are model 
parameters. The \code{"active"} field (logical) determines whether the 
parameters are estimated (\code{TRUE}) or remain fixed (\code{FALSE}).
\end{ldescription}
\end{Arguments}
\begin{Details}\relax
Restoration algorithm:  \eqn{ P = P_{min} + (P_{max} - P_{min}) (sin(\frac{\pi S}{2}))^2 }{}
\end{Details}
\begin{Value}
Parameter vector converted from scaled units to original units specified by \code{pvec}.
\end{Value}
\begin{Author}\relax
Jon Schnute, Pacific Biological Station, Nanaimo BC
\end{Author}
\begin{SeeAlso}\relax
\code{\LinkA{scalePar}{scalePar}}, \code{\LinkA{calcMin}{calcMin}}, \code{\LinkA{GT0}{GT0}}
\end{SeeAlso}
\begin{Examples}
\begin{ExampleCode}
pvec <- data.frame(val=c(1,100,10000),min=c(0,0,0),max=c(5,500,50000),
        active=c(TRUE,TRUE,TRUE))
S    <- c(.5,.5,.5)
P    <- restorePar(S,pvec)
print(cbind(pvec,S,P))
\end{ExampleCode}
\end{Examples}

\HeaderA{runDemos}{Interactive GUI for R Demos}{runDemos}
\keyword{utilities}{runDemos}
\begin{Description}\relax
An interactive GUI for accessing demos from any R package installed on the 
user's system. \code{runDemos} is a convenient alternative to R's \code{demo} 
function.
\end{Description}
\begin{Usage}
\begin{verbatim}runDemos(package)\end{verbatim}
\end{Usage}
\begin{Arguments}
\begin{ldescription}
\item[\code{package}] display demos from a particular package (optional).
\end{ldescription}
\end{Arguments}
\begin{Details}\relax
If the argument \code{package} is not specified, the function will look for 
demos in all packages installed on the user's system.
\end{Details}
\begin{Note}\relax
The \code{runDemos} GUI attempts to retain the user's objects and restore
the working directory. However, pre-existing objects will be overwritten 
if their names coincide with names used by the various demos. Also, 
depending on conditions, the user may lose working directory focus. 
We suggest that cautious users run this demo from a project where data objects are 
not critical.
\end{Note}
\begin{Author}\relax
Alex Couture-Beil, Malaspina University-College, Nanaimo BC
\end{Author}
\begin{SeeAlso}\relax
\code{\LinkA{runExamples}{runExamples}}  for examples specific to \pkg{PBSmodelling}.
\end{SeeAlso}

\HeaderA{runExamples}{Run GUI Examples Included with PBS Modelling}{runExamples}
\keyword{utilities}{runExamples}
\begin{Description}\relax
Display an interactive GUI to demonstrate PBS Modelling examples.

The example source files can be found in the R directory 
\code{.../library/PBSmodelling/examples}.
\end{Description}
\begin{Usage}
\begin{verbatim}runExamples()\end{verbatim}
\end{Usage}
\begin{Details}\relax
Some examples use external packages which must be installed to work correctly:

\code{BRugs} - \code{LinReg}, \code{MarkRec}, and \code{CCA};

\code{odesolve/ddesolve} - \code{FishRes};

\code{PBSmapping} - \code{FishTows}.
\end{Details}
\begin{Note}\relax
The examples are copied from \code{.../library/PBSmodelling/examples} to R's current 
temporary working directory and run from there.
\end{Note}
\begin{Author}\relax
Alex Couture-Beil, Malaspina University-College, Nanaimo BC
\end{Author}
\begin{SeeAlso}\relax
\code{\LinkA{runDemos}{runDemos}}
\end{SeeAlso}

\HeaderA{scalePar}{Scale Parameters to [0,1]}{scalePar}
\keyword{optimize}{scalePar}
\begin{Description}\relax
Scale parameters for function minimization by \code{calcMin}.
\end{Description}
\begin{Usage}
\begin{verbatim}scalePar(pvec)\end{verbatim}
\end{Usage}
\begin{Arguments}
\begin{ldescription}
\item[\code{pvec}] a data frame comprising four columns - 
\code{c("val","min","max","active")} and as many rows as there are model
parameters. The \code{"active"} field (logical) determines whether the 
parameters are estimated (\code{TRUE}) or remain fixed (\code{FALSE}).
\end{ldescription}
\end{Arguments}
\begin{Details}\relax
Scaling algorithm:  \eqn{ S = \frac{2}{\pi} asin \sqrt{ \frac{P - P_{min}}{P_{max} - P_{min}} } }{}
\end{Details}
\begin{Value}
Parameter vector scaled between 0 and 1.
\end{Value}
\begin{Author}\relax
Jon Schnute, Pacific Biological Station, Nanaimo BC
\end{Author}
\begin{SeeAlso}\relax
\code{\LinkA{restorePar}{restorePar}}, \code{\LinkA{calcMin}{calcMin}}, \code{\LinkA{GT0}{GT0}}
\end{SeeAlso}
\begin{Examples}
\begin{ExampleCode}
pvec <- data.frame(val=c(1,100,10000),min=c(0,0,0),max=c(5,500,50000),
        active=c(TRUE,TRUE,TRUE))
S    <- scalePar(pvec)
print(cbind(pvec,S))
\end{ExampleCode}
\end{Examples}

\HeaderA{setFileOption}{Set a PBS File Path Option Interactively}{setFileOption}
\begin{Description}\relax
Set a PBS option by browsing for a file. This function provides 
an alternative to using \code{setPBSoptions} when setting an 
option that has a path to a file as its value.
\end{Description}
\begin{Usage}
\begin{verbatim}
setFileOption(option)
\end{verbatim}
\end{Usage}
\begin{Arguments}
\begin{ldescription}
\item[\code{option}] name PBS option to change
\end{ldescription}
\end{Arguments}
\begin{Note}\relax
If all the required arguments are missing, it is assumed that 
the function is being called by a GUI widget.
\end{Note}
\begin{Author}\relax
Anisa Egeli, Vancouver Island University, Nanaimo BC
\end{Author}
\begin{SeeAlso}\relax
\code{\LinkA{setPathOption}{setPathOption}}, \code{\LinkA{setPBSoptions}{setPBSoptions}}
\end{SeeAlso}
\begin{Examples}
\begin{ExampleCode}
## Not run: 
setPathOption("editor")
## End(Not run)
\end{ExampleCode}
\end{Examples}

\HeaderA{setGUIoptions}{Set PBS Options from Widget Values}{setGUIoptions}
\begin{Description}\relax
Set PBS options from corresponding values of widgets in a GUI.
\end{Description}
\begin{Usage}
\begin{verbatim}
setGUIoptions(option)
\end{verbatim}
\end{Usage}
\begin{Arguments}
\begin{ldescription}
\item[\code{option}] the name of a single option or the string \code{"*"}.
\end{ldescription}
\end{Arguments}
\begin{Details}\relax
A GUI may have PBS options that it uses, which have corresponding widgets that
are used for entering values for these options. These are declared by
\code{declareGUIoptions}.

If the \code{option} argument is the name of an option, 
\code{setGUIoptions} transfers the value of this option from a 
same-named widget into PBS options global R environment database.

If the \code{option} argument is \code{"*"}, then all the 
options that have been declared by \code{declareGUIoptions} 
will be transferred in this fashion.

To use this function in a \emph{window description file}, the 
\code{option} argument must be specified as the action of the 
widget that calls \code{setGUIoptions} -- \code{action=editor} 
or \code{action=*} for example.
\end{Details}
\begin{Note}\relax
If all the required arguments are missing, it is assumed that 
the function is being called by a GUI widget.
\end{Note}
\begin{Author}\relax
Anisa Egeli, Vancouver Island University, Nanaimo BC
\end{Author}
\begin{SeeAlso}\relax
\code{\LinkA{declareGUIoptions}{declareGUIoptions}}, \code{\LinkA{getGUIoptions}{getGUIoptions}},
\code{\LinkA{setPBSoptions}{setPBSoptions}},
\end{SeeAlso}
\begin{Examples}
\begin{ExampleCode}
## Not run: 
setGUIoptions("editor")
## End(Not run)
\end{ExampleCode}
\end{Examples}

\HeaderA{setPathOption}{Set a PBS Path Option Interactively}{setPathOption}
\begin{Description}\relax
Set a PBS option by browsing for a directory. This function provides 
an alternative to using \code{setPBSoptions} when setting an option 
that has a path as its value.
\end{Description}
\begin{Usage}
\begin{verbatim}
setPathOption(option)
\end{verbatim}
\end{Usage}
\begin{Arguments}
\begin{ldescription}
\item[\code{option}] name PBS option to change
\end{ldescription}
\end{Arguments}
\begin{Note}\relax
If all the required arguments are missing, it is assumed that 
the function is being called by a GUI widget.
\end{Note}
\begin{Author}\relax
Anisa Egeli, Vancouver Island University, Nanaimo BC
\end{Author}
\begin{SeeAlso}\relax
\code{\LinkA{setFileOption}{setFileOption}}, \code{\LinkA{setPBSoptions}{setPBSoptions}}
\end{SeeAlso}
\begin{Examples}
\begin{ExampleCode}
## Not run: 
setPathOption("myPath")
## End(Not run)
\end{ExampleCode}
\end{Examples}

\HeaderA{setPBSext}{Set a Command Associated with a File Name Extension}{setPBSext}
\keyword{methods}{setPBSext}
\begin{Description}\relax
Set a command with an associated extension, for use in 
\code{openFile}.  The command must specify where the target file 
name is inserted by indicating a \code{"\%f"}.
\end{Description}
\begin{Usage}
\begin{verbatim}
setPBSext(ext, cmd)
\end{verbatim}
\end{Usage}
\begin{Arguments}
\begin{ldescription}
\item[\code{ext}] string specifying the extension suffix.
\item[\code{cmd}] command string to associate with the extension.
\end{ldescription}
\end{Arguments}
\begin{Note}\relax
These values are not saved from one \emph{PBS Modelling} session to the next.
\end{Note}
\begin{Author}\relax
Alex Couture-Beil, Malaspina University-College, Nanaimo BC
\end{Author}
\begin{SeeAlso}\relax
\code{\LinkA{getPBSext}{getPBSext}}, \code{\LinkA{openFile}{openFile}}, \code{\LinkA{clearPBSext}{clearPBSext}}
\end{SeeAlso}

\HeaderA{setPBSoptions}{Set A User Option}{setPBSoptions}
\keyword{methods}{setPBSoptions}
\begin{Description}\relax
Options set by the user for use by other functions.
\end{Description}
\begin{Usage}
\begin{verbatim}
setPBSoptions(option, value, sublist=FALSE)
\end{verbatim}
\end{Usage}
\begin{Arguments}
\begin{ldescription}
\item[\code{option}] name of the option to set.
\item[\code{value}] new value to assign this option.
\item[\code{sublist}] if \code{value} is a sublist (list component) of \code{option},
this list component can be changed individually using \code{sublist=TRUE}.
\end{ldescription}
\end{Arguments}
\begin{Note}\relax
A value \code{.PBSmod\$.options\$.optionsChanged} is set to \code{TRUE} when an option is changed,
so that the user doesn't always have to be prompted to save the options file. \\
By default, \code{.PBSmod\$.options\$.optionsChanged} is not set or \code{NULL}. \\
Also, if an option is set to \code{""} or \code{NULL} then it is removed. \\
\code{.initPBSoptions()} is now called first (options starting with a dot "." 
do not set \code{.optionsChanged}).
\end{Note}
\begin{Author}\relax
Alex Couture-Beil, Malaspina University-College, Nanaimo BC
\end{Author}
\begin{SeeAlso}\relax
\code{\LinkA{getPBSoptions}{getPBSoptions}}, \code{\LinkA{writePBSoptions}{writePBSoptions}},
\code{\LinkA{readPBSoptions}{readPBSoptions}}
\end{SeeAlso}

\HeaderA{setwdGUI}{Browse for Working Directory and Optionally Find Prefix}{setwdGUI}
\begin{Description}\relax
Allows the user to browse a directory tree to set the working directory.
Optionally, files with given suffixes can be located in the new directory.
\end{Description}
\begin{Usage}
\begin{verbatim}
setwdGUI(suffix)
\end{verbatim}
\end{Usage}
\begin{Arguments}
\begin{ldescription}
\item[\code{suffix}] character vector of suffixes or \code{""} (See Details).
\end{ldescription}
\end{Arguments}
\begin{Details}\relax
The \code{suffix} argument is passed to a call to 
\code{findPrefix} after the working directory is changed 
(See \code{setwd}). If \code{suffix} is set to the empty
string \code{""}, then \code{findPrefix} will not be called.

To use this function in a \emph{window description file}, 
the \code{suffix} argument must be specified as the action of 
the widget that calls \code{setwdGUI}. Furthermore, the
suffixes must be separated by commas (e.g., \code{action=.c,.cpp}). 
If \code{action=,} is specified, then \code{findPrefix} will not be called.
\end{Details}
\begin{Value}
If suffixes are given, a character vector of prefixes of all files in
the working directory that match one of the given suffixes is returned; 
otherwise, the function returns \code{invisible()}.
\end{Value}
\begin{Note}\relax
If all the required arguments are missing, it is assumed that the function
is being called by a GUI widget.
\end{Note}
\begin{Author}\relax
Anisa Egeli, Vancouver Island University, Nanaimo BC
\end{Author}
\begin{SeeAlso}\relax
\code{\LinkA{findPrefix}{findPrefix}}, \code{\LinkA{setwd}{setwd}}
\end{SeeAlso}
\begin{Examples}
\begin{ExampleCode}
## Not run: 
#match files that end with ".a" followed by 0 or more characters, ".b" followed
#by any single character, ".c", or "-old.d" (a suffix does not have to be a
#file extension)
findPrefix(".a*", ".b?", ".c", "-old.d")
## End(Not run)
\end{ExampleCode}
\end{Examples}

\HeaderA{setWidgetState}{Update Widget State}{setWidgetState}
\keyword{methods}{setWidgetState}
\begin{Description}\relax
Update the read-only state of a widget.
\end{Description}
\begin{Usage}
\begin{verbatim}
setWidgetState( varname, state, radiovalue, winname )
\end{verbatim}
\end{Usage}
\begin{Arguments}
\begin{ldescription}
\item[\code{varname}] the name of the widget
\item[\code{state}] "normal" or "disabled"; entry and text widgets also support "readonly"
\item[\code{radiovalue}] if specified, disable a particular radio option, as identified by the value, rather than the complete set (identified by the common name)
\item[\code{winname}] window from which to select the GUI widget. The default 
takes the window that has most recently received new user input.
\end{ldescription}
\end{Arguments}
\begin{Details}\relax
The \code{varname} argument expects a name which corresponds to some widget with the same corresponding name value.
Alternatively, any element can be updated by appending its index in square brackets 
to the end of the \code{name}. The \code{data} widget is indexed differently 
than the \code{matrix} widget by adding "d" after the brackets. This tweak is necessary 
for the internal coding (bookkeeping) of \emph{PBS Modelling}. Example: \code{"foo[1,1]d"}.

The state can either be "normal" which allows the user to edit values, or "disabled" which
restricts the user from editing the values. Entry widgets also support "readonly" which will
allow the user to copy and paste data.
\end{Details}
\begin{Author}\relax
Alex Couture-Beil
\end{Author}
\begin{Examples}
\begin{ExampleCode}
## Not run: 
winDesc <- c('vector length=3 name=vec labels="normal disabled readonly" values="1 2 3"',
             "matrix nrow=2 ncol=2 name=mat", "button name=but_name" );
createWin(winDesc, astext=TRUE)

setWidgetState( "vec[1]", "normal" )
setWidgetState( "vec[2]", "disabled" )
setWidgetState( "vec[3]", "readonly" )

setWidgetState( "mat", "readonly" )
setWinVal( list( mat = matrix( 1:4, 2, 2 ) ) )

#works for buttons too
setWidgetState( "but_name", "disabled" )
## End(Not run)
\end{ExampleCode}
\end{Examples}

\HeaderA{setWinAct}{Add a Window Action to the Saved Action Vector}{setWinAct}
\keyword{methods}{setWinAct}
\begin{Description}\relax
Append a string value specifying an action to the first position of an 
action vector.
\end{Description}
\begin{Usage}
\begin{verbatim}
setWinAct(winName, action)
\end{verbatim}
\end{Usage}
\begin{Arguments}
\begin{ldescription}
\item[\code{winName}] window name where action is taking place.
\item[\code{action}] string value describing an action.
\end{ldescription}
\end{Arguments}
\begin{Details}\relax
When a function is called from a GUI, a string descriptor associated with 
the action of the function is stored internally (appended to the first position 
of the action vector). A user can utilize this action as a type of argument 
for programming purposes. The command \code{getWinAct()[1]} yields the latest action.

Sometimes it is useful to \dQuote{fake} an action. Calling \code{setWinAct} allows 
the recording of an action, even if a button has not been pressed.
\end{Details}
\begin{Author}\relax
Alex Couture-Beil, Malaspina University-College, Nanaimo BC
\end{Author}

\HeaderA{setWinVal}{Update Widget Values}{setWinVal}
\keyword{methods}{setWinVal}
\begin{Description}\relax
Update a widget with a new value.
\end{Description}
\begin{Usage}
\begin{verbatim}
setWinVal(vars, winName)
\end{verbatim}
\end{Usage}
\begin{Arguments}
\begin{ldescription}
\item[\code{vars}] a list or vector with named components.
\item[\code{winName}] window from which to select GUI widget values. The default 
takes the window that has most recently received new user input.
\end{ldescription}
\end{Arguments}
\begin{Details}\relax
The \code{vars} argument expects a list or vector with named elements. 
Every element name corresponds to the widget name which will be updated 
with the supplied element value.

The \code{vector}, \code{matrix}, and \code{data} widgets can be updated in 
several ways. If more than one name is specified for the \code{names} argument 
of these widgets, each element is treated like an \code{entry} widget. 

If however, a single \code{name} describes any of these three widgets, the entire 
widget can be updated by passing an appropriately sized object.

Alternatively, any element can be updated by appending its index in square brackets 
to the end of the \code{name}. The \code{data} widget is indexed differently 
than the \code{matrix} widget by adding "d" after the brackets. This tweak is necessary 
for the internal coding (bookkeeping) of \emph{PBS Modelling}. Example: \code{"foo[1,1]d"}.
\end{Details}
\begin{Author}\relax
Alex Couture-Beil, Malaspina University-College, Nanaimo BC
\end{Author}
\begin{SeeAlso}\relax
\code{\LinkA{getWinVal}{getWinVal}}, \code{\LinkA{createWin}{createWin}}
\end{SeeAlso}
\begin{Examples}
\begin{ExampleCode}
## Not run: 
winDesc <- c("vector length=3 name=vec",
             "matrix nrow=2 ncol=2 name=mat",
             "slideplus name=foo");
createWin(winDesc, astext=TRUE)
setWinVal(list(vec=1:3, "mat[1,1]"=123, foo.max=1.5, foo.min=0.25, foo=0.7))
## End(Not run)
\end{ExampleCode}
\end{Examples}

\HeaderA{show0}{Convert Numbers into Text with Specified Decimal Places}{show0}
\keyword{print}{show0}
\begin{Description}\relax
Return a character representation of a number with added zeroes 
out to a specified number of decimal places.
\end{Description}
\begin{Usage}
\begin{verbatim}
show0(x, n, add2int = FALSE)
\end{verbatim}
\end{Usage}
\begin{Arguments}
\begin{ldescription}
\item[\code{x}] numeric data (scalar, vector, or matrix).
\item[\code{n}] number of decimal places to show, including zeroes.
\item[\code{add2int}] If \code{TRUE}, add zeroes on the end of integers.
\end{ldescription}
\end{Arguments}
\begin{Value}
A scalar/vector of strings representing numbers. Useful for labelling purposes.
\end{Value}
\begin{Note}\relax
This function does not round or truncate numbers. It simply adds zeroes if 
\code{n} is greater than the available digits in the decimal part of a number.
\end{Note}
\begin{Author}\relax
Rowan Haigh, Pacific Biological Station, Nanaimo BC
\end{Author}
\begin{Examples}
\begin{ExampleCode}
frame()

#do not show decimals on integers
addLabel(0.25,0.75,show0(15.2,4))
addLabel(0.25,0.7,show0(15.1,4))
addLabel(0.25,0.65,show0(15,4))

#show decimals on integers
addLabel(0.25,0.55,show0(15.2,4,TRUE))
addLabel(0.25,0.5,show0(15.1,4,TRUE))
addLabel(0.25,0.45,show0(15,4,TRUE))
\end{ExampleCode}
\end{Examples}

\HeaderA{showAlert}{Display a Message in an Alert Window}{showAlert}
\begin{Description}\relax
Display an alert window that contains a specified message and 
an OK button for dismissing the window.
\end{Description}
\begin{Usage}
\begin{verbatim}
showAlert(message, title="Alert", icon="warning")
\end{verbatim}
\end{Usage}
\begin{Arguments}
\begin{ldescription}
\item[\code{message}] message to display in alert window
\item[\code{title}] title of alert window
\item[\code{icon}] icon to display in alert window; options are 
\code{"error"}, \code{"info"}, \code{"question"}, or \code{"warning"}.
\end{ldescription}
\end{Arguments}
\begin{Author}\relax
Anisa Egeli, Vancouver Island University, Nanaimo BC
\end{Author}
\begin{SeeAlso}\relax
\code{\LinkA{getYes}{getYes}}
\end{SeeAlso}
\begin{Examples}
\begin{ExampleCode}
## Not run: 
showAlert("Hello World!")
## End(Not run)
\end{ExampleCode}
\end{Examples}

\HeaderA{showArgs}{Display Expected Widget Arguments}{showArgs}
\keyword{utilities}{showArgs}
\keyword{character}{showArgs}
\begin{Description}\relax
For each widget specified, display its arguments in order with their default values. 
The display list can be expanded to report each argument on a single line.
\end{Description}
\begin{Usage}
\begin{verbatim}
showArgs(widget, width=70, showargs=FALSE)
\end{verbatim}
\end{Usage}
\begin{Arguments}
\begin{ldescription}
\item[\code{widget}] vector string of widget names; if not specified (default), 
the function displays information about all widgets in alphabetical order.
\item[\code{width}] numeric width used by \code{strwrap} to wrap lines of the widget
usage section.
\item[\code{showargs}] logical:, if \code{TRUE}, the display also lists each argument
on single line after the widget usage section.
\end{ldescription}
\end{Arguments}
\begin{Value}
A text stream to the R console. Invisibly returns the widget usage lines.
\end{Value}
\begin{Author}\relax
Alex Couture-Beil, Malaspina University-College, Nanaimo BC
\end{Author}

\HeaderA{showHelp}{Display Help Pages for Packages in HTML Browser}{showHelp}
\keyword{device}{showHelp}
\keyword{utilities}{showHelp}
\begin{Description}\relax
Display the help pages for installed packages that match the 
supplied pattern in an HTML browser window.
\end{Description}
\begin{Usage}
\begin{verbatim}
showHelp(pat="methods")
\end{verbatim}
\end{Usage}
\begin{Arguments}
\begin{ldescription}
\item[\code{pat}] string pattern to match to package names
\end{ldescription}
\end{Arguments}
\begin{Details}\relax
The specified pattern is matched to R-packages installed on 
the user's system. The code uses the \code{PBSmodelling} 
function \code{openFile} to display the HTML Help Pages using 
a program that the system associates with \code{html} 
extensions. On systems that do not support file extension 
associations, the function \code{setPBSext} can temporarily 
set a command to associate with an extension.
\end{Details}
\begin{Value}
A list is invisibly returned, comprising:
\begin{ldescription}
\item[\code{Apacks}] all packages installed on user's system
\item[\code{Spacks}] selected packages based on specified pattern
\item[\code{URLs}] path and file name of HTML Help Page
\end{ldescription}

Help pages are displayed in a separate browser window.
\end{Value}
\begin{Note}\relax
The connection time for browsers (at least in Windows OS)
is slow. If the HTML browser program is not already running,
multiple matching pages will most likely not be displayed. However, 
subsequent calls to \code{showHelp} should show all matches.
\end{Note}
\begin{Author}\relax
Rowan Haigh, Pacific Biological Station, Nanaimo BC
\end{Author}
\begin{SeeAlso}\relax
\code{\LinkA{openFile}{openFile}}, \code{\LinkA{setPBSext}{setPBSext}}, \code{\LinkA{getPBSext}{getPBSext}}
\end{SeeAlso}

\HeaderA{showPacks}{Show Packages Required But Not Installed}{showPacks}
\keyword{package}{showPacks}
\keyword{character}{showPacks}
\begin{Description}\relax
Show the packages specified by the user and compare these to the 
installed packages on the user's system. Display packages not installed.
\end{Description}
\begin{Usage}
\begin{verbatim}
showPacks(packs=c("PBSmodelling","PBSmapping","PBSddesolve",
    "rgl","deSolve","akima","deldir","sp","maptools","KernSmooth"))
\end{verbatim}
\end{Usage}
\begin{Arguments}
\begin{ldescription}
\item[\code{packs}] string vector of package names that are compared to installed packages. 
\end{ldescription}
\end{Arguments}
\begin{Value}
Invisibly returns a list of \code{Apacks} (all packages installed on user's system),
\code{Ipacks} (packages in \code{packs} that are installed), and
\code{Mpacks} (packages that are missing).
\end{Value}
\begin{Author}\relax
Jon Schnute, Pacific Biological Station, Nanaimo BC
\end{Author}

\HeaderA{showRes}{Show Results of Expression Represented by Text}{showRes}
\keyword{utilities}{showRes}
\begin{Description}\relax
Evaluate the supplied expression, reflect it on the command line, 
and show the results of the evaluation.
\end{Description}
\begin{Usage}
\begin{verbatim}
showRes(x, cr=TRUE, pau=TRUE)
\end{verbatim}
\end{Usage}
\begin{Arguments}
\begin{ldescription}
\item[\code{x}] an R expression to evaluate
\item[\code{cr}] logical: if \code{TRUE}, introduce extra carriage returns
\item[\code{pau}] logical: if \code{TRUE}, pause after expression reflection and execution
\end{ldescription}
\end{Arguments}
\begin{Value}
The results of the expression are return invisibly.
\end{Value}
\begin{Author}\relax
Jon Schnute, Pacific Biological Station, Nanaimo BC
\end{Author}
\begin{Examples}
\begin{ExampleCode}
showRes("x=rnorm(100)",pau=FALSE)
\end{ExampleCode}
\end{Examples}

\HeaderA{showVignettes}{Display Vignettes for Packages}{showVignettes}
\keyword{utilities}{showVignettes}
\begin{Description}\relax
Create a GUI that displays all vignettes for installed packages. 
The user can choose to view the source file for building the vignette 
or the final \code{.pdf} file.
\end{Description}
\begin{Usage}
\begin{verbatim}
showVignettes(package)
\end{verbatim}
\end{Usage}
\begin{Arguments}
\begin{ldescription}
\item[\code{package}] character string specifying package name that exists in the user's R library
\end{ldescription}
\end{Arguments}
\begin{Details}\relax
If the argument \code{package} is not specified, the function
will look for vignettes in all packages installed on the user's 
system. The user can choose to view the source file 
for building the vignette (usually \code{*.Rnw} or \code{*.Snw} files)
or the final build from the source code (\code{*.pdf}).

\code{showVignettes} uses the \pkg{PBSmodelling} function 
\code{openFile} to display the \code{.Rnw} and \code{.pdf} files 
using programs that the system associates with these extensions. 
On systems that do not support file extension associations, the 
function \code{setPBSext} can temporarily set a command to associate 
with an extension.
\end{Details}
\begin{Author}\relax
Anisa Egeli, Vancouver Island University, Nanaimo BC
\end{Author}
\begin{SeeAlso}\relax
\code{\LinkA{showHelp}{showHelp}}, \code{\LinkA{openFile}{openFile}}, \code{\LinkA{setPBSext}{setPBSext}}, \code{\LinkA{getPBSext}{getPBSext}}
\end{SeeAlso}

\HeaderA{sortHistory}{Sort an Active or Saved History}{sortHistory}
\keyword{list}{sortHistory}
\begin{Description}\relax
Utility to sort history. When called without any arguments, an interactive GUI
is used to pick which history to sort. When called with \code{hisname}, sort
this active history widget. When called with \code{file} and \code{outfile},
sort the history located in \code{file} and save to \code{outfile}.
\end{Description}
\begin{Usage}
\begin{verbatim}
sortHistory(file="", outfile=file, hisname="")
\end{verbatim}
\end{Usage}
\begin{Arguments}
\begin{ldescription}
\item[\code{file}] file name of saved history to sort.
\item[\code{outfile}] file to save sorted history to.
\item[\code{hisname}] name of active history widget and window it is located in, given
in the form \code{WINDOW.HISTORY}.
\end{ldescription}
\end{Arguments}
\begin{Details}\relax
After selecting a history to sort (either from given arguments, or interactive GUI)
the R data editor window will be displayed. The editor will have one column named
\"new\" which will have numbers 1,2,3,...,n. This represents the current ordering
of the history. You may change the numbers around to define a new order. The list
is sorted by reassigning the index in row i as index i.

For example, if the history had three items 1,2,3. Reordering this to 3,2,1 will
reverse the order; changing the list to 1,2,1,1 will remove entry 3 and 
create two duplicates of entry 1.
\end{Details}
\begin{Author}\relax
Alex Couture-Beil, Malaspina University-College, Nanaimo BC
\end{Author}
\begin{SeeAlso}\relax
\code{\LinkA{importHistory}{importHistory}}, \code{\LinkA{initHistory}{initHistory}}
\end{SeeAlso}

\HeaderA{testAlpha}{Test Various Alpha Transparency Values}{testAlpha}
\keyword{color}{testAlpha}
\begin{Description}\relax
Display how the alpha transparency for \code{rgb()} varies.
\end{Description}
\begin{Usage}
\begin{verbatim}
testAlpha(alpha=seq(0,1,len=25), fg="blue", bg="yellow",
      border="black", grid=FALSE, ...)
\end{verbatim}
\end{Usage}
\begin{Arguments}
\begin{ldescription}
\item[\code{alpha}] numeric vector of alpha transparency values values from 0 to 1. 
\item[\code{fg}] foreground colour of the top shape that varies in transparency. 
\item[\code{bg}] background colour (remains constant) of the underlying shape. 
\item[\code{border}] border colour (which also changes in transparency) of the foreground polygon. 
\item[\code{grid}] logical: if \code{TRUE}, lay a grey grid on the background colour. 
\item[\code{...}] additional graphical arguments to send to the the ploting functions. 
\end{ldescription}
\end{Arguments}
\begin{Value}
Invisibly returns the compound RGB matrix for \code{fg}, \code{alpha}, 
\code{bg}, and \code{border}.
\end{Value}
\begin{Author}\relax
Jon Schnute, Pacific Biological Station, Nanaimo BC
\end{Author}
\begin{SeeAlso}\relax
\code{\LinkA{testCol}{testCol}}, \code{\LinkA{testPch}{testPch}}, \code{\LinkA{testLty}{testLty}}, \code{\LinkA{testLwd}{testLwd}}
\end{SeeAlso}

\HeaderA{testCol}{Display Named Colours Available Based on a Set of Strings}{testCol}
\keyword{utilities}{testCol}
\keyword{color}{testCol}
\begin{Description}\relax
Display colours as patches in a plot. Useful for programming purposes. 
Colours can be specified in any of 3 different ways: (i) by colour name, 
(ii) by hexidecimal colour code created by \code{rgb()}, or (iii) by an 
index to the \code{color()} palette.
\end{Description}
\begin{Usage}
\begin{verbatim}
testCol(cnam=colors()[sample(length(colors()),15)])
\end{verbatim}
\end{Usage}
\begin{Arguments}
\begin{ldescription}
\item[\code{cnam}] vector of colour names to display. Defaults to 15 random names 
from the \code{color} palette.
\end{ldescription}
\end{Arguments}
\begin{Author}\relax
Rowan Haigh, Pacific Biological Station, Nanaimo BC
\end{Author}
\begin{SeeAlso}\relax
\code{\LinkA{pickCol}{pickCol}}
\end{SeeAlso}
\begin{Examples}
\begin{ExampleCode}
testCol(c("sky","fire","sea","wood"))

testCol(c("plum","tomato","olive","peach","honeydew"))

testCol(substring(rainbow(63),1,7))

#display all colours set in the colour palette
testCol(1:length(palette()))

#they can even be mixed
testCol(c("#9e7ad3", "purple", 6))
\end{ExampleCode}
\end{Examples}

\HeaderA{testLty}{Display Line Types Available}{testLty}
\keyword{utilities}{testLty}
\keyword{color}{testLty}
\begin{Description}\relax
Display line types available.
\end{Description}
\begin{Usage}
\begin{verbatim}
testLty(newframe = TRUE)
\end{verbatim}
\end{Usage}
\begin{Arguments}
\begin{ldescription}
\item[\code{newframe}] if \code{TRUE}, create a new blank frame, otherwise overlay current frame.
\end{ldescription}
\end{Arguments}
\begin{Note}\relax
Quick representation of first 20 line types for reference purposes.
\end{Note}
\begin{Author}\relax
Rowan Haigh, Pacific Biological Station, Nanaimo BC
\end{Author}

\HeaderA{testLwd}{Display Line Widths}{testLwd}
\keyword{utilities}{testLwd}
\keyword{color}{testLwd}
\begin{Description}\relax
Display line widths. User can specify particular ranges for \code{lwd}. 
Colours can also be specified and are internally repeated as necessary.
\end{Description}
\begin{Usage}
\begin{verbatim}
testLwd(lwd=1:20, col=c("black","blue"), newframe=TRUE)
\end{verbatim}
\end{Usage}
\begin{Arguments}
\begin{ldescription}
\item[\code{lwd}] line widths to display. Ranges can be specified.
\item[\code{col}] colours to use for lines. Patterns are repeated if 
\code{length(lwd) > length(col)}
\item[\code{newframe}] if \code{TRUE}, create a new blank frame, otherwise overlay current frame.
\end{ldescription}
\end{Arguments}
\begin{Author}\relax
Rowan Haigh, Pacific Biological Station, Nanaimo BC
\end{Author}
\begin{Examples}
\begin{ExampleCode}
testLwd(3:15,col=c("salmon","aquamarine","gold"))
\end{ExampleCode}
\end{Examples}

\HeaderA{testPch}{Display Plotting Symbols and Backslash Characters}{testPch}
\keyword{utilities}{testPch}
\keyword{color}{testPch}
\begin{Description}\relax
Display plotting symbols. User can specify particular ranges (increasing 
continuous integer) for \code{pch}.
\end{Description}
\begin{Usage}
\begin{verbatim}
testPch(pch=1:100, ncol=10, grid=TRUE, newframe=TRUE, bs=FALSE)
\end{verbatim}
\end{Usage}
\begin{Arguments}
\begin{ldescription}
\item[\code{pch}] symbol codes to view.
\item[\code{ncol}] number of columns in display (can only be 2, 5, or 10). Most 
sensibly this is set to 10.
\item[\code{grid}] logical: if \code{TRUE}, grid lines are plotted for visual aid.
\item[\code{newframe}] logical: if \code{TRUE} reset the graph, otherwise overlay 
on top of the current graph.
\item[\code{bs}] logical: if \code{TRUE}, show backslash characters used in text 
statements (e.g., \code{30\bsl{}272C} = 30\eqn{^\circ}{}C).
\end{ldescription}
\end{Arguments}
\begin{Author}\relax
Rowan Haigh, Pacific Biological Station, Nanaimo BC
\end{Author}
\begin{Examples}
\begin{ExampleCode}
testPch(123:255)
testPch(1:25,ncol=5)
testPch(41:277,bs=TRUE)
\end{ExampleCode}
\end{Examples}

\HeaderA{testWidgets}{Display Sample GUIs and their Source Code}{testWidgets}
\aliasA{widgets}{testWidgets}{widgets}
\keyword{utilities}{testWidgets}
\begin{Description}\relax
Display an interactive GUI to demonstrate the available widgets in PBS Modelling.
A \code{text} window displays the \emph{window description file} source code. The user
can modify this sample code and recreate the test GUI by pressing the button below.

The \emph{Window Description Files} can be found in the R directory \\
\code{.../library/PBSmodelling/testWidgets}.
\end{Description}
\begin{Usage}
\begin{verbatim}testWidgets()\end{verbatim}
\end{Usage}
\begin{Details}\relax
Following are the widgets and default values supported by PBS Modelling.
For detailed descriptions, see Appendix A in \sQuote{PBSModelling-UG.pdf} 
located in the R directory \code{.../library/PBSmodelling/doc}.

\begin{alltt}
button text="Calculate" font="" fg="black" bg="" width=0 name=NULL
   function="" action="button" sticky="" padx=0 pady=0

check name mode="logical" checked=FALSE text="" font="" fg="black"
   bg="" function="" action="check" edit=TRUE sticky="" padx=0 pady=0

data nrow ncol names modes="numeric" rowlabels="" collabels=""
   rownames="X" colnames="Y" font="" fg="black" bg="" entryfont=""
   entryfg="black" entrybg="white" noeditfg="black" noeditbg="gray"
   values="" byrow=TRUE function="" enter=TRUE action="data"
   edit=TRUE width=6 borderwidth=0 sticky="" padx=0 pady=0

droplist name values=NULL choices=NULL labels=NULL selected=1
   add=FALSE font="" fg="black" bg="white" function="" enter=TRUE
   action="droplist" edit=TRUE mode="character" width=20 sticky=""
   padx=0 pady=0

entry name value="" width=20 label=NULL font="" fg="" bg=""
   entryfont="" entryfg="black" entrybg="white" noeditfg="black"
   noeditbg="gray" edit=TRUE password=FALSE function="" enter=TRUE
   action="entry" mode="numeric" sticky="" padx=0 pady=0

grid nrow=1 ncol=1 toptitle="" sidetitle="" topfont="" sidefont=""
   topfg=NULL sidefg=NULL fg="black" topbg=NULL sidebg=NULL bg=""
   byrow=TRUE borderwidth=1 relief="flat" sticky="" padx=0 pady=0

history name="default" function="" import="" fg="black" bg=""
   entryfg="black" entrybg="white" text=NULL textsize=0 sticky=""
   padx=0 pady=0

include file=NULL name=NULL

label text="" name="" mode="character" font="" fg="black" bg=""
   sticky="" justify="left" wraplength=0 width=0 padx=0 pady=0

matrix nrow ncol names rowlabels="" collabels="" rownames=""
   colnames="" font="" fg="black" bg="" entryfont="" entryfg="black"
   entrybg="white" noeditfg="black" noeditbg="gray" values=""
   byrow=TRUE function="" enter=TRUE action="matrix" edit=TRUE
   mode="numeric" width=6 borderwidth=0 sticky="" padx=0 pady=0

menu nitems=1 label font="" fg="" bg=""

menuitem label font="" fg="" bg="" function action="menuitem"

null bg="" padx=0 pady=0

object name rowshow=0 font="" fg="black" bg="" entryfont=""
   entryfg="black" entrybg="white" noeditfg="black" noeditbg="gray"
   vertical=FALSE collabels=TRUE rowlabels=TRUE function=""
   enter=TRUE action="data" edit=TRUE width=6 borderwidth=0 sticky=""
   padx=0 pady=0

radio name value text="" font="" fg="black" bg="" function=""
   action="radio" edit=TRUE mode="numeric" selected=FALSE sticky=""
   padx=0 pady=0

slide name from=0 to=100 value=NA showvalue=FALSE
   orientation="horizontal" font="" fg="black" bg="" function=""
   action="slide" sticky="" padx=0 pady=0

slideplus name from=0 to=1 by=0.01 value=NA font="" fg="black" bg=""
   entryfont="" entryfg="black" entrybg="white" function=""
   enter=FALSE action="slideplus" sticky="" padx=0 pady=0

spinbox name from to by=1 value=NA label="" font="" fg="black" bg=""
   entryfont="" entryfg="black" entrybg="white" function=""
   enter=TRUE edit=TRUE action="droplist" width=20 sticky="" padx=0
   pady=0

table name rowshow=0 font="" fg="black" bg="white" rowlabels=""
   collabels="" function="" action="table" edit=TRUE width=10
   sticky="" padx=0 pady=0

text name height=8 width=30 edit=FALSE scrollbar=TRUE fg="black"
   bg="white" mode="character" font="" value="" borderwidth=1
   relief="sunken" sticky="" padx=0 pady=0

vector names length=0 labels="" values="" vecnames="" font=""
   fg="black" bg="" entryfont="" entryfg="black" entrybg="white"
   noeditfg="black" noeditbg="gray" vertical=FALSE function=""
   enter=TRUE action="vector" edit=TRUE mode="numeric" width=6
   borderwidth=0 sticky="" padx=0 pady=0

window name="window" title="" vertical=TRUE bg="\#D4D0C8" fg="\#000000"
   onclose="" remove=FALSE

\end{alltt}
\end{Details}
\begin{Author}\relax
Alex Couture-Beil, Malaspina University-College, Nanaimo BC
\end{Author}
\begin{SeeAlso}\relax
\code{\LinkA{createWin}{createWin}}, \code{\LinkA{showArgs}{showArgs}}
\end{SeeAlso}

\HeaderA{unpackList}{Unpack List Elements into Variables}{unpackList}
\keyword{file}{unpackList}
\keyword{list}{unpackList}
\begin{Description}\relax
Make local or global variables (depending on the scope specified) from 
the named components of a list.
\end{Description}
\begin{Usage}
\begin{verbatim}
unpackList(x, scope="L")
\end{verbatim}
\end{Usage}
\begin{Arguments}
\begin{ldescription}
\item[\code{x}] named list to unpack.
\item[\code{scope}] If \code{"L"}, create variables local to the parent frame 
that called the function. If \code{"G"}, create global variables.
\end{ldescription}
\end{Arguments}
\begin{Value}
A character vector of unpacked variable names.
\end{Value}
\begin{Author}\relax
Alex Couture-Beil, Malaspina University-College, Nanaimo BC
\end{Author}
\begin{SeeAlso}\relax
\code{\LinkA{packList}{packList}}, \code{\LinkA{readList}{readList}}, \code{\LinkA{writeList}{writeList}}
\end{SeeAlso}
\begin{Examples}
\begin{ExampleCode}
x <- list(a=21,b=23);
unpackList(x);
print(a);
\end{ExampleCode}
\end{Examples}

\HeaderA{updateGUI}{Update Active GUI With Local Values}{updateGUI}
\keyword{methods}{updateGUI}
\begin{Description}\relax
Update the currently active GUI with values from R's memory at the specified location.
\end{Description}
\begin{Usage}
\begin{verbatim}
updateGUI(scope = "L")
\end{verbatim}
\end{Usage}
\begin{Arguments}
\begin{ldescription}
\item[\code{scope}] either \code{"L"} for the parent frame, \code{"G"} for the global environment,
or an explicit R environment
\end{ldescription}
\end{Arguments}
\begin{Details}\relax
If the characteristics of the local R objects do not match those of the GUI objects,
the update will fail.
\end{Details}
\begin{Value}
Invisibly returns a Boolean vector that specifies whether the objects in the local R
environment match items in the active GUI.
\end{Value}
\begin{Author}\relax
Rob Kronlund, Pacific Biological Station, Nanaimo BC
\end{Author}
\begin{SeeAlso}\relax
\code{\LinkA{getWinVal}{getWinVal}}, \code{\LinkA{setWinVal}{setWinVal}}
\end{SeeAlso}

\HeaderA{vbdata}{Data: Lengths-at-Age for von Bertalanffy Curve}{vbdata}
\keyword{datasets}{vbdata}
\begin{Description}\relax
Lengths-at-age for freshwater mussels (\emph{Anodonta kennerlyi}).
\end{Description}
\begin{Usage}
\begin{verbatim}data(vbdata)\end{verbatim}
\end{Usage}
\begin{Format}\relax
A data frame with 16 rows and 2 columns \code{c("age","len")}.
\end{Format}
\begin{Details}\relax
Data for demonstration of the von Bertalanffy model used in the 
\code{\LinkA{calcMin}{calcMin}} example.
\end{Details}
\begin{Source}\relax
Fisheries and Oceans Canada - Mittertreiner and Schnute (1985)
\end{Source}
\begin{References}\relax
Mittertreiner, A. and Schnute, J. (1985) Simplex: a manual and software package 
for easy nonlinear parameter estimation and interpretation in fishery research. 
\emph{Canadian Technical Report of Fisheries and Aquatic Sciences} \bold{1384}, xi + 90 pp.
\end{References}

\HeaderA{vbpars}{Data: Initial Parameters for a von Bertalanffy Curve}{vbpars}
\keyword{datasets}{vbpars}
\begin{Description}\relax
Starting parameter values for \code{Linf}, \code{K}, and \code{t0} for 
von Bertalanffy minimization using length-at-age data (\code{\LinkA{vbdata}{vbdata}}) 
for freshwater mussels (\emph{Anodonta kennerlyi}).
\end{Description}
\begin{Usage}
\begin{verbatim}data(vbpars)\end{verbatim}
\end{Usage}
\begin{Format}\relax
A matrix with 3 rows and 3 columns \code{c("Linf","K","t0")}. Each row contains 
the starting values, minima, and maxima, respectively, for the three parameters.
\end{Format}
\begin{Details}\relax
Data for demonstration of the von Bertalanffy model used in 
the \code{\LinkA{calcMin}{calcMin}} example.
\end{Details}
\begin{References}\relax
Mittertreiner, A. and Schnute, J. (1985) Simplex: a manual and software package 
for easy nonlinear parameter estimation and interpretation in fishery research. 
\emph{Canadian Technical Report of Fisheries and Aquatic Sciences} \bold{1384}, xi + 90 pp.
\end{References}

\HeaderA{view}{View First/Last/Random n Elements/Rows of an Object}{view}
\keyword{print}{view}
\begin{Description}\relax
View the first or last or random \code{n} elements or rows of an object. 
Components of lists will be subset also.
\end{Description}
\begin{Usage}
\begin{verbatim}
view(obj, n=5, last=FALSE, random=FALSE, ...)
\end{verbatim}
\end{Usage}
\begin{Arguments}
\begin{ldescription}
\item[\code{obj}] object to view.
\item[\code{n}] first (default)/last/random \code{n} elements/rows of \code{obj} to view.
\item[\code{last}] logical: if \code{TRUE}, last \code{n} elements/rows of \code{obj} are displayed.
\item[\code{random}] logical: if \code{TRUE}, \code{n} random elements/rows 
(without replacement) of \code{obj} are displayed.
\item[\code{...}] additional arguments (e.g., \code{replace=T} if specifying \code{random=T}).
\end{ldescription}
\end{Arguments}
\begin{Author}\relax
Rowan Haigh, Pacific Biological Station, Nanaimo BC
\end{Author}

\HeaderA{viewCode}{View Package R Code}{viewCode}
\keyword{character}{viewCode}
\keyword{package}{viewCode}
\begin{Description}\relax
View the R code of all functions in a specified package 
installed on the user's system.
\end{Description}
\begin{Usage}
\begin{verbatim}
viewCode(pkg="PBSmodelling", funs)
\end{verbatim}
\end{Usage}
\begin{Arguments}
\begin{ldescription}
\item[\code{pkg}] string name of a package installed on the user's computer. 
\item[\code{funs}] string vector of explicit function names from \code{pkg} to view. 
\end{ldescription}
\end{Arguments}
\begin{Details}\relax
If \code{funs} is not specified, then all functions, including
hidden (dot) functions are displayed. \\
If the package has a namespace, functions there can also be displayed.
\end{Details}
\begin{Value}
Invisibly returns source code of all functions in the specified package.
The function invokes \code{openFile} to display the results.
\end{Value}
\begin{Author}\relax
Rowan Haigh, Pacific Biological Station, Nanaimo BC
\end{Author}

\HeaderA{writeList}{Write a List to a File in PBS Modelling Format}{writeList}
\keyword{file}{writeList}
\keyword{list}{writeList}
\begin{Description}\relax
Write an ASCII text representation in either \code{"D"} format or \code{"P"} format. 
The \code{"D"} format makes use of \code{dput} and \code{dget}, and produces an R 
representation of the list. The \code{"P"} format represents a simple list in an 
easy-to-read, ad hoc \code{PBSmodelling} format.
\end{Description}
\begin{Usage}
\begin{verbatim}
writeList(x, fname, format="D", comments="")
\end{verbatim}
\end{Usage}
\begin{Arguments}
\begin{ldescription}
\item[\code{x}] R list object to write to an ASCII text file.
\item[\code{fname}] file name of the text file containing the list.
\item[\code{format}] format of the file to create: \code{"D"} or \code{"P"}.
\item[\code{comments}] vector of character strings to use as initial-line comments in the file.
\end{ldescription}
\end{Arguments}
\begin{Details}\relax
The \code{"D"} format is equivalent to using R's \code{base} functions 
\code{dput} and \code{dget}, which support all R objects.

The \code{"P"} format only supports named lists of vectors, matrices, arrays, and data frames. 
Scalars are treated like vectors. Nested lists are not supported. 

The \code{"P"} format writes each named element in a list using the 
following conventions: (i) \$ followed by the name of the data object 
to denote the start of that object's description; (ii) \$\$ on the next line 
to describe the object's structure - object type, mode(s), names (if vector), 
rownames (if matrix or data), and colnames (if matrix or data); and 
(iii) subsequent lines of data (one line for vector, multiple lines for matrix or data).

Arrays with three or more dimensions have dim and dimnames arguments. Dim is the dimension
of the data, a vector as returned by \code{dim(some\_array)}, and dimnames is a vector of length
\code{sum(dim(some\_array)+1)} and is constructed as follows:

\code{
                foreach dimension d
                        first append the name of the dimension d
                        then append all labels within that dimension
        }

Multiple rows of data for matrices or data frames must have equal 
numbers of entries (separated by whitespace).

Using \code{"P"} formatting, array data are written the same way that 
they are displayed in the R console: \\
\code{nrow=dim()[1]}, \code{ncol=dim()[2]} \\
repeated by scrolling through successively higher dimensions, increasing the 
index from left to right within each dimension. The flattened table will have 
\code{dim()[2]} columns.

For complete details, see \dQuote{PBSmodelling-UG.pdf} in the 
R directory \code{.../library/PBSmodelling/doc}.
\end{Details}
\begin{Author}\relax
Alex Couture-Beil, Malaspina University-College, Nanaimo BC
\end{Author}
\begin{SeeAlso}\relax
\code{\LinkA{packList}{packList}}, \code{\LinkA{readList}{readList}}, \code{\LinkA{unpackList}{unpackList}}
\end{SeeAlso}
\begin{Examples}
\begin{ExampleCode}
## Not run: 
test <- list(a=10,b=euro,c=view(WorldPhones),d=view(USArrests))
writeList(test,"test.txt",format="P",
        comments=" Scalar, Vector, Matrix, Data Frame")
openFile("test.txt")
## End(Not run)

##Example of dimnames for Arrays
dimnames(Titanic)
writeList( list( Titanic ), format="P")
\end{ExampleCode}
\end{Examples}

\HeaderA{writePBSoptions}{Write PBS Options to an External File}{writePBSoptions}
\begin{Description}\relax
Save options that were set using \code{setPBSoptions},
\code{setPBSext}, or interfaces such as \code{loadC}.  
These options can be reloaded using \code{readPBSoptions}.
\end{Description}
\begin{Usage}
\begin{verbatim}
writePBSoptions(fname="PBSoptions.txt")
\end{verbatim}
\end{Usage}
\begin{Arguments}
\begin{ldescription}
\item[\code{fname}] file name or full path of file to which the options will be saved.
\end{ldescription}
\end{Arguments}
\begin{Note}\relax
Options with names starting with \code{"."} will not be saved.
\end{Note}
\begin{Author}\relax
Anisa Egeli, Vancouver Island University, Nanaimo BC
\end{Author}
\begin{SeeAlso}\relax
\code{\LinkA{readPBSoptions}{readPBSoptions}}, \code{\LinkA{setPBSoptions}{setPBSoptions}},
\code{\LinkA{setPBSext}{setPBSext}}, \code{\LinkA{promptWriteOptions}{promptWriteOptions}}
\end{SeeAlso}

\printindex{}
\end{document}
